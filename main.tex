\documentclass[a4paper, 12pt, openright, oneside, final]{book}%{scrbook}
\usepackage{eso-pic}
\usepackage{float}
\usepackage{graphicx}
\usepackage{setspace}
\usepackage[english]{babel}
\usepackage{listings}
\usepackage{varioref}
\usepackage{subfigure}
\usepackage{verbatim}
\usepackage{amsmath}
\usepackage[pdftex,bookmarks=true,hypertexnames=true]{hyperref}
\usepackage{fancyhdr}
\newcommand{\fncyblank}{\fancyhf{}}
\newcommand{\HRule}{\rule{\linewidth}{0.5mm}}
\usepackage{lettrine}
\newenvironment{abstract}%
{\cleardoublepage\fncyblank\null\vfill\begin{center}%
\bfseries\abstractname\end{center}}%
{\vfill\null}

%\lstset{%
%  basicstyle=\small,
%  frame=tb,
%  captionpos=b,
%  stringstyle=\ttfamily,
%  showstringspaces=false,
%  numbers=left,
%  numberstyle=\tiny,
%  framextopmargin=2pt,
%  framexbottommargin=2pt,
%  stepnumber=1,
%  numbersep=5pt}

\usepackage{courier}
\usepackage{color}
\usepackage{xcolor}
\lstset{
         basicstyle=\footnotesize\ttfamily, % Standardschrift
         %numbers=left,               % Ort der Zeilennummern
         numberstyle=\tiny,          % Stil der Zeilennummern
         %stepnumber=2,               % Abstand zwischen den Zeilennummern
         numbersep=5pt,              % Abstand der Nummern zum Text
         tabsize=2,                  % Groesse von Tabs
         extendedchars=true,         %
         breaklines=true,            % Zeilen werden Umgebrochen
         keywordstyle=\color{red},
 %        frame=b,         
 %        keywordstyle=[1]\textbf,    % Stil der Keywords
 %        keywordstyle=[2]\textbf,    %
 %        keywordstyle=[3]\textbf,    %
 %        keywordstyle=[4]\textbf,   \sqrt{\sqrt{}} %
         stringstyle=\color{black}\ttfamily, % Farbe der String
         showspaces=false,           % Leerzeichen anzeigen ?
         showtabs=false,             % Tabs anzeigen ?
         xleftmargin=5pt,
         framexleftmargin=5pt,
         framexrightmargin=5pt,
         framexbottommargin=4pt,
         %backgroundcolor=\color{lightgray},
         showstringspaces=false      % Leerzeichen in Strings anzeigen ?        
}
\lstloadlanguages{% Check Dokumentation for further languages ...
         %[Visual]Basic
         %Pascal
         C
         %C++
         %XML
         %HTML
         %Java
}
%\DeclareCaptionFont{blue}{\color{blue}} 

\usepackage{caption} \DeclareCaptionFont{white}{\color{white}}
\DeclareCaptionFormat{listing}{\colorbox[cmyk]{0.43, 0.35,
0.35,0.01}{\parbox{\textwidth}{\hspace{15pt}#1#2#3}}}
\captionsetup[lstlisting]{format=listing,labelfont=white,textfont=white,
singlelinecheck=false, margin=0pt, font={bf,footnotesize}}

\hypersetup{
pdfauthor= {Sun Youcheng},
pdftitle= {Design and Development of Real-Time Multi-Processor Bandwidth
Control Mechanisms in General Purpose Operating Systems},
pdfsubject= {Design and Development of Real-Time Multi-Processor Bandwidth
Control Mechanisms in General Purpose Operating Systems},
pdfkeywords = {realtime, scheduling, linux, multiprocessor, hierarchy},
pdfborder = { 0 0 0 0 }
}

%
% TODO: it is better to use title and author keywords here
%
\title{Design and Development of Real-Time Multi-Processor Bandwidth
  Control Mechanisms in General Purpose Operating Systems}

\author{Youcheng Sun}

\onehalfspacing

\begin{document}

\frontmatter

%\maketitle

\thispagestyle{empty}
\begin{center}
  \AddToShipoutPicture*{\AtPageCenter{\makebox(0,0){\includegraphics{images/trento_sssup}}}}
  \textsc{\Large Universit\`a di Pisa}
  \vskip3mm
  \textsc{\large Facolt\`a di Ingegneria}
  \vskip3mm
  \textsc{\normalsize Corso di Laurea Specialistica in Ingegneria Informatica}
  \vskip5mm
  \textsc{\large Tesi di Laurea Specialistica}
  \vskip22mm
  \vfill
  \textsc{\textbf{\huge Design and development of deadline based scheduling
mechanisms for multiprocessor systems\\}}
  \vfill
  \vskip22mm

  \begin{tabular*}{\textwidth}{@{\extracolsep{\fill}}lcr}
    Relatori: 				            & \hfill & Candidato: \\
    & & \\
    Prof. Giuseppe Lipari                & \hfill & Juri Lelli \\
    \emph{\scriptsize Scuola Superiore Sant'Anna } & \hfill & \\
    & & \\
    & & \\
    Prof. Giuseppe Anastasi   & \hfill & \\
    \emph{\scriptsize Dipartimento di Ingegneria dell'Informazione }        & \hfill & \\
  \end{tabular*}
  \vskip12mm
  \textsc{\normalsize Anno Accademico 2009/2010}
\end{center}

\cleardoublepage
\thispagestyle{empty}
\null\vspace{2cm}
%\stretch{0.5}
\begin{flushright}%\emph{
	\small
	%Alla mia famiglia\\
	%e a Claudia.}
\end{flushright}
\vspace{\stretch{2}}\null

\begin{abstract}
  % TODO: the abstract should be a short summary of the contents of
  % the thesis. Very short motivation (the problem to be solved), and
  % what has been done in the thesis.
  % DONE

  %Attentions have been being paid to extend general purpose operating
  %systems with real time functionalities. In this thesis, a scheduling
  %framework, which can control Central Processing Unit (CPU) bandwidth
  %distribution on multi-processor platforms in a real-time way, is
  %proposed in Linux.  Under the framework, cpu bandwidths from
  %different processors are reserved according to Constant Bandwidth
  %Server (CBS) rules.  Yet as for how to utilize the reserved
  %bandwidths to schedule tasks in Linux, this is not the interest of
  %the framework. They can be scheduled by policies used in the Linux
  %system or scheduling algorithms that are implemented for a specific
  %purpose. A subset of these tasks can get a portion of the
  %reservation using the same rule.  Furthermore, under the framework,
  %scheduling policies can be applied in a controlled scale instead of
  %the whole system.

  %In current implementation, both normal tasks and real-time tasks in
  %Linux can work under the framework. Experimental results are not
  %available now ...
  Attentions have been being paid to extend general purpose operating
  systems (GPOS) with real time functionalities. In this thesis, the
  Open-Extension Container(OXC) scheduling framework, which can control 
  Central Processing Unit (CPU) bandwidth distribution on multi-processor 
  platforms in a real-time way, is proposed in Linux. The ox container 
  is a new data structure defined in our work. It has a good feature 
  that from a Linux scheduler's aspect, it behaves as a virtual CPU.  
  Under the framework, CPU bandwidths from different processors are 
  reserved according to Constant Bandwidth Server (CBS) rules.  Yet 
  as for how to utilize the reserved bandwidths to schedule tasks, 
  this is not the interest of the framework. They can be scheduled by 
  policies used in the Linux system or scheduling algorithms that are 
  implemented for a specific purpose. A subset of these tasks can get 
  a portion of the reservation using the same rule.  Furthermore, under 
  the framework, scheduling policies can be applied in a controlled scale 
  instead of the whole system.

  In current implementation, both normal tasks and real-time tasks in
  Linux can work under the framework. Experiments show that our 
  hierarchical CPU bandwidth control is feasible in Linux.
\end{abstract}

\tableofcontents
\listoffigures
\lstlistoflistings
\chapter{Introduction\label{chap:introduction}}

Linux is the most widely deployed open source general purpose
operating system. Traditionally, the aim of Linux scheduling is to
distribute cpu cycles fairly among tasks and task groups according to
their relative importances. Because of the popularity and
compatibility, Linux based systems are used to serve various kinds of
purposes. In some cases, fairness is not enough or necessary.

There is the class of \emph{real time} applications 
that have soft or hard requirements on the cpu cycles they receive,
normally in some time length, in order to work well. Such timing
guarantee (or real time guarantee) cannot be provided just by fairness. 
Multi-media applications belong to this class. Indeed, simply from cpu 
bandwidth management's point of view, a real time gurantee is also 
useful. In many situations, people prefer to distributing cpu power in 
a privileged and predictable way. For instance, to give 10\% of the 
total cpu cycles to a set of tasks; furthermore, take half from the 10\% 
and assign it to a subset of the tasks.

Nowadays, multi-core architecture is successfully utilized to boost 
computing capability for computing devices. It's no surprise to see an 
embedded device with more than one CPU inside. Computing systems with 
multiple processors are becoming mainstream. Yet high speed processors 
and more cores do bring more challenges instead of solving the CPU timing 
guarantee problem. Also, no matter how powerful the platform is, there is 
always time when people need more. So, to fully utilize multi-core platform 
is also a good reason to manage computing power in a real time way. 

In mainline Linux kernel, there are two so called real time scheduling 
policies SCHED\_FIFO and SCHED\_RR. Tasks scheduled by them are called 
''real time(rt) tasks''. They are required in POSIX standard for 
POSIX-compliant operating systems like Linux. Unfortunately, despite of 
the name, these two policies can only provide real time gurantee in very 
limited conditions. In mainline Linux, there exist two non real time 
mechanisms to control CPU bandwidth distribution: real time(rt) throttling 
and complete fairness scheduler(CFS) bandwidth control. In principle, they 
are the same technique working for different types of tasks.

There is work that extends Linux with real time capabilities to fullfill 
the timing guarantee requirement: RTAI\cite{rtai}, AQuoSA\cite{Luigi09}, 
\texttt{sched-deadline} patch\cite{Dario09}, 
IRMOS real time framework, and RESCH and so on.
Instead of modifying the system directly, RT-Xen tries to apply real 
time mechanisms in the hypervisor level(Xen). Each work has its emphasis. 
Initially, our work is motivated by IRMOS. Our framework has two features:
\begin{itemize}
\item 	It can predictively distribute cpu cycles above multi-processor 
	platform to a set of tasks and its subsets, without requirements 
	for details how they are scheduled. 
\item 	Under the framework, scheduling policies can be applied in a
	fine-grained way.
\end{itemize}

In Linux, there is a scheduling system on each CPU. Different such per
CPU scheduling systems construct the system level scheduling by task
migration mechanisms among different CPUs. For the first time in Linux, 
our framework provides with the opportunity to build extra scheduling 
systems besides these destinated with each CPU. We call the framework 
Open-Extension Container(OXC) scheduling framework. Open-Extension (OX) 
container is a new data structure in Linux that is raised by our work. 
It is the fundamental element in the framework. Based on ox container 
structure, the concept ''per oxc scheduling system'', whose 
behaviour is the same as ''per CPU scheduling system'' in Linux, is 
given. Several per oxc scheduling systems coorperate and work as the 
''pseudo (Linux) system level scheduling''.  

In oxc scheduling framework, each ox-container can reserve an amount of
bandwidth from a CPU through CBS rules \cite{AbeniB98}. The per oxc 
scheduling system based on it utilize this computing power to shcedule 
tasks as if working on a less powerful cpu. This is how oxc framework 
distribute reserved CPU cycles to tasks under it. Because the per oxc 
scheduling has the same behaviour as per CPU scheduling in Linux, 
general types of tasks can run using the reserved bandwidth. On multiple
processor platforms, different OXCs can inpdependently reserve bandwidths 
from a subset of total CPUs and scheduling systems above them work together 
to imitate the behaviour of the Linux system level scheduling. The basic 
unit to apply a scheduling policy under OXC framework is an OX container.

\mainmatter
\chapter{Background\label{chap:background}}
\section{The Constant Bandwidth Server\label{sec:CBS}}
A Constant Bandwidth Server(CBS) is characterized by a budget $c_s$ and
an ordered pair $(Q_s, T_s)$, where $Q_s$ is the maximum budget and $T_s$
is the period of the server. $U_s = Q_s/T_s$ is called the server bandwidth.
Such a server can be utilized to serve a set of
tasks, which can be non real time or (hard and soft) real time tasks. 
CBS defines rules to reserve bandwidth on a single processor.
In our work, we use a hard version of CBS.

For a specific server $S$, at any instant, a fixed deadline $d_{s,k}$ is 
associated with the server. A CBS is said to be active at time $t$ if there 
are pending tasks and $c_s$ is not 0; otherwise it is called idle. 
At any time, among all active servers,
the one with earliest deadline is chosen. Then a served task of this server 
is picked to execute. CBS does not restrict the rule to pick up a particular 
task. For example, first in first out (fifo), rate monotonic scheduling and
any user defined rule can be used.
As the picked task executes, the server budget $c_s$ is decreased by the 
same amount. When budget $c_s$ reaches 0, the server become inactive. At 
each deadline point, the $c_s$ will be recharged to $Q_s$ and a new server 
deadline will be generated as $d_{s, k+1} = d_{s,k} + T_k$. Initially, 
$c_s = Q_s$ and $d_{s, 0} = 0$. When a task arrives at time $t$ and the 
server is idle, if $c_s \ge (d_{s,k} - t)U_s$, the server updates its 
deadline as $d_{s, k+1} = t + T_s$ and $c_s$ is recharged to maximum 
value $Q_s$.

Given a set of servers $\{S_0, S_1, ... , S_n\}$, if
\[
	\sum_{i=0}^n U_i \le 1
\]
then, every $T_i$ time units, a server $S_i$ can obtain $Q_s$ time units 
to serve its tasks. In other words, $U_i$ is the bandwidth a server $S_i$
reserves from a cpu.

\section{The Linux Scheduler\label{sec:LinuxSched}}
A scheduler is responsible for distributing CPU cycles to tasks in the system
according to some scheduling algorithm. In Linux, tasks refer to a process or 
a thread and correspond to the data structure \texttt{struct task\_struct}.

\subsection{Scheduling classes\label{sec:LinuxSched_classes}}
Linux scheduling system adapts a modular design, and the basic modularity is
a scheduling class, which is an instance of \texttt{struct sched\_class}. 
The \texttt{struct sched\_class} defines a set of interfaces which need to be 
realized in order to implement a scheduler in Linux. 
There are three scheduling classes in mainline Linux: rt\_sched\_class,
cfs\_sched\_class and idle\_sched\_class. Each scheduling class is 
responsible for scheduling a type of tasks. Tasks scheduled
\texttt{cfs\_sched\_class} are called normal tasks. Tasks scheduled
by \texttt{rt\_sched\_class} are called rt tasks. 
\texttt{idle\_sched\_class} deals with special idles tasks which
does nothing and occupy the CPU when no rt or normal tasks need
a CPU.
Each scheduler fullfill
details behind the interface. Let's have a look at the 
\texttt{struct sched\_class} structure:
\begin{lstlisting}
struct sched_class {
	const struct sched_class *next;
	void (*enqueue_task) (struct rq *rq, struct task_struct *p, int flags);
	void (*dequeue_task) (struct rq *rq, struct task_struct *p, int flags);
	void (*check_preempt_curr) (struct rq *rq, struct task_struct *p, int flags);
	struct task_struct * (*pick_next_task) (struct rq *rq);
	void (*task_tick) (struct rq *rq, struct task_struct *p, int queued);
	...
};
\end{lstlisting}
\begin{itemize} 
\item \texttt{next:}
	Scheduling classes are linked in a chain. Whenever a task is needed 
	to pick up, the scheduler from the beginning to the end of the chain 
	is checked until a task is found, as shown in \ref{fig:sched_classes}. 
	So, schedulers in front have higher priority. 
\item \texttt{enqueue\_task:}
	Called when a task enters a runnable state. It puts the scheduling 
	entity (task or task group) into the runqueue structure, 
	i.e. \texttt{struct rq}.
\item \texttt{dequeue\_task:}
	When a task is no longer runnable, this function is called to move
	corresponding scheduling entity out of the runqueue.
\item \texttt{check\_preempt\_curr:}
	This function checks if a task that entered the runnable state 
	should preempt the currently running task.
\item \texttt{pick\_next\_task:}
	This function chooses the task to run next.
\item \texttt{task\_tick:}
	This function is the most frequently called function. 
	It is called every time tick in the system, it might lead to task
	switch.
\end{itemize} 
\begin{figure}[htbp]
        \centering
        \includegraphics[width=\textwidth]{images/sched_classes}
        \caption{Scheduling classes in Linux}
        \label{fig:sched_classes}
\end{figure}
The basic scheduling unit in Linux is scheduling entity, which can represent
tasks or task groups. There are two kinds of scheduling entities:
\texttt{struct sched\_entity} for cfs scheduling class and 
\texttt{struct sched\_rt\_entity} for rt scheduling class.
When \texttt{CONFIG\_FAIR\_GROUP\_SCHED} is set, cfs task grouping 
is enabled. And \texttt{CONFIG\_RT\_GROUP\_SCHEED} is the kernel configuration
for rt task group scheduling.
\begin{lstlisting}
struct task_struct {
	...
	struct sched_entity se;
	struct sched_rt_entity rt;
	...
};

struct task_group {
#ifdef CONFIG_FAIR_GROUP_SCHED
	/* sched_entity of this group on each cpu */
	struct sched_entity **se;
	...
#endif
#ifdef CONFIG_RT_GROUP_SCHED
	/* sched_rt_entity of this group on each cpu */
	struct sched_entity **rt_se;
	...
#endif
	...
};
\end{lstlisting}

\subsection{Runqueue centered scheduling\label{LinuxSched_rq}}
Every hook in \texttt{struct sched\_class} deals with the data structure 
\texttt{struct rq}, which is called run queue in Linux.  We can say that 
Linux scheduling is runqueue centered. In Linux, the \texttt{struct rq} is 
a per CPU data structure; each cpu is associated with a runqueue. Although 
the name indicates, \texttt{struct rq} is not a queue. Let's have a look at 
the inside of a runqueue structure:
\begin{lstlisting}
struct rq {
	...
	unsigned long nr_running;
	struct cfs_rq cfs;
	struct rt_rq rt;
	struct task_struct *curr, *idle;
	u64 clock;
#ifdef CONFIG_SMP
	int cpu;
#endif
	...
};
\end{lstlisting}
\begin{itemize}
\item \texttt{nr\_running} specifies the number of runnable tasks in the
	runqueue.
\item \texttt{cfs and rt} are two specific runqueues for 
	\texttt{cfs\_sched\_class} and \texttt{rt\_sched\_class} respectively. 
	In order to handle specific type of tasks, different schedulers define 
	new type of runqueue data structures. For example, \texttt{struct cfs 
	rq} and \texttt{struct rt\_rq}. 
\item \texttt{curr} points to the task currently running.
\item \texttt{idle} points to a special idle task when no other tasks are
	runnable.
\item \texttt{clock} is updated by \texttt{update\_rq\_clock} method.
\item \texttt{cpu} tells the CPU of this runqueue.
\end{itemize}

\subsection{Completely Fair scheduler\label{sec:LinuxSched_cfs}}
Completely fair scheduler is implemented in \texttt{fair\_sched\_class}. 
Most tasks inside Linux are scheduled by completely fair scheduling class
 and are normal tasks, which can be further divided into three sub 
types given scheduling policies (\texttt{SCHED\_NORMAL}, \texttt{SCHED\_BATCH}
and \texttt{SCHED\_IDLE\footnote{This SCHED\_IDLE policy is not related to
idle\_sched\_class which aims to handle a special idle task.}}).

CFS tries to distribute CPU cycles fairly to tasks and task groups according
to their \emph{weight}. A specific runqueue structure \texttt{struct cfs\_rq}
is provided to deal with normal tasks. Recall an instance of such cfs runqueue
is embedded in the per CPU runqueue and each task group holds a pointer
to cfs runqueue on each CPU to store tasks belonging to it.
The scheduling entity handled by cfs scheduling class is 
\texttt{struct sched\_entity}. Here instead of studying the details of
CFS, we are going to see how different scheduling components
(\texttt{sched\_entity}, \texttt{task\_struct}, \texttt{task\_group}
and \texttt{struct cfs\_rq}) are related.
\begin{lstlisting}
struct cfs_rq {
        unsigned long nr_running;
        u64 min_vruntime;
        struct rb_root tasks_timeline;
#ifdef CONFIG_FAIR_GROUP_SCHED
        struct rq *rq;  
	struct task_group *tg;
#endif
	...
};
\end{lstlisting}
\begin{itemize}
\item \texttt{nr\_running} is the number of tasks in this cfs runqueue.
\item \texttt{min\_vruntime} tracks the minimum virtual runtime of all
	tasks associated with this cfs runqueue.
\item \texttt{tasks\_timeline} is the root of the red-black tree where 
	all tasks associated with this cfs runqueue is stored. Tasks are 
	ordered by their virtual run time value.
\item \texttt{rq} is the per CPU runqueue this cfs runqueue is embedded in.
\item \texttt{tg} is the task group that owns this cfs runqueue.
\end{itemize}
\begin{lstlisting}
struct sched_entity {
	...
	struct cfs_rq *cfs_rq;
#ifdef CONFIG_FAIR_GROUP_SCHED
	struct cfs_rq *my_q;
#endif
	...
}; 
\end{lstlisting}
\begin{itemize}
\item \texttt{cfs\_rq} is where this entity is to be queued.
\item \texttt{my\_rq} is the cfs runqueue owned by this entity(group).
	Remember that a schedulign entity can also represent a task group.
\end{itemize}
When cfs task group scheduling is enabled. In this case, the cfs scheduling 
scheme is shown in figure\ref{fig:cfs_scheme_tg}. This is not a complete
scheme: 1) Under a task group there could be sub groups, which behave as 
the task in the figure 2) In the system, there is a top group, which includes
all tasks in the system defaultly; tasks in this group are enqueued in the
cfs runqueue embedded in the per CPU runqueue directly.
\begin{figure}[htbp]
        \centering
        \includegraphics[width=\textwidth]{images/cfs_scheduling_scheme_tg}
        \caption{CFS scheduling when cfs group scheduling is enabled.}
        \label{fig:cfs_scheme_tg}
\end{figure}

If cfs task group scheduling is not enabled, a task is directed to its 
per CPU runqueue by a \emph{task\_rq} marco. \emph{task\_rq} also works 
for rt tasks when rt task group scheduling is not enabled.
\ref{fig:sched_scheme_no_tg}.
\begin{figure}[htbp]
        \centering
        \includegraphics[height=0.1\textheight,width=0.5\textwidth]{images/scheduling_scheme_no_tg}
        \caption{Scheduling scheme without group scheduling}
        \label{fig:sched_scheme_no_tg}
\end{figure}

\subsection{Real time scheduler\label{LinuxSched_rt}}
Tasks with POSIX real time policies \texttt{SCHED\_FIFO} and \texttt{SCHED\_RR}
are scheduled by the real time scheduling class \texttt{rt\_sched\_class} and
are called rt tasks. Given figure\ref{fig:sched_classes}, rt tasks are always
schedueld over normal tasks. 

\texttt{SCHED\_FIFO} implements a simple first-in, first-out scheduling 
algorithm. A running \texttt{SCHED\_FIFO} task can only be preempted by a 
higher priority rt task. \texttt{SCHED\_RR} is \texttt{SCHED\_FIFO} with 
timeslices --- it is a round robin algrithm. When a \texttt{SCHED\_RR}
task exhausts its timeslice, another \texttt{SCHED\_RR} task of the same
priprity is picked to run a timeslice, and so on. In either case, a rt task
cannot be preempted by a lower priority task.

The rt scheduling class provides with a sub runqueue structure 
\texttt{struct rt\_rq} to deal with rt tasks.
\begin{lstlisting}
struct rt_rq {
	struct rt_prio_array active;
        unsigned long rt_nr_running;
#ifdef CONFIG_RT_GROUP_SCHED
        struct rq *rq;
        struct task_group *tg;
#endif
	...
};

struct rt_prio_array {
	DECLARE_BITMAP(bitmap, MAX_RT_PRIO+1); 
	struct list_head queue[MAX_RT_PRIO];
};
\end{lstlisting}
All rt tasks with the same priority are kept in a linked list headed by
$active.queue[prio]$. If there is a task in the liset, the corresponding 
bit in $active.bitmap$ is set.

The connections among \texttt{struct sched\_rt\_entity} and other scheduling
components are similar to the \text{struct sched\_entity} case.
\begin{lstlisting}
struct sched_rt_entity {
	...
	struct rt_rq *rt_rq;
#ifdef CONFIG_RT_GROUP_SCHED
	struct rt_rq *my_q;
#endif
	...
}; 
\end{lstlisting}
When \texttt{CONFIG\_RT\_GROUP\_SCHED} is set, figure\ref{fig:rt_scheme_tg} 
shows the scheduling scheme of rt tasks. If rt task group scheduling is 
not enabled, still \emph{task\_rq} marco will be used.
\begin{figure}[htbp]
        \centering
        \includegraphics[width=\textwidth]{images/rt_scheduling_scheme_tg}
        \caption{RT scheduling when rt group scheduling is enabled.}
        \label{fig:rt_scheme_tg}
\end{figure}

\section{Related work}
\subsection{rt-throttling}
\subsection{cfs bandwidth control}
\subsection{AQuoSA}
\subsection{schedule-deadline}
\subsection{RESCH}
\subsection{IRMOS real-time framework}

\chapter{Design of oxc scheduling framework}


\chapter{Development of OXC Framework\label{chap:impl}}
The oxc framework is still ongoing. Latest codes can be found 
in github\footnote{https://github.com/YIYAYIYAYOUCHENG/linux}.
The oxc framework is not a scheduler. Although it cooperates with
different scheduling classes, its pure responsibility is managing 
the distribution of CPU power, which depends on modular schedulers 
to use it for scheduling tasks. Under oxc framework, we also call
oxc scheduling as oxc control since it is utilized to control CPU 
bandwidth reservation.

\section{Implementation of ox container structure}
An ox container \texttt{struct oxc\_rq}, listing~\ref{lst:oxc_rq}, 
is defined in Linux kernel. In \texttt{struct oxc\_rq} there 
are fields for reserving bandwidth from a CPU using CBS rules. 
In the following context, an \texttt{struct oxc\_rq} instance is 
also called an ox container, container, or oxc runqueue for the 
same meaning. An oxc runqueue also corresponds to a constant 
bandwidth server in CBS theory. 
The CPU reservation for oxc runqueues follows the implementation 
of CBS reservation for a RT runqueue in IRMOS framework. 

\begin{lstlisting}[language=C, 
		caption={The ox container : \texttt{struct oxc\_rq}},
                        label={lst:oxc_rq}]
struct oxc_rq {
	unsigned long oxc_nr_running;
	inx oxc_throttled;
	u64 oxc_deadline;
	u64 oxc_time;
	u64 oxc_runtime;
	ktime_t oxc_period;
	struct hrtimer oxc_period_timer;
	raw_spin_lock oxc_runtime_lock;
	struct rq rq_;
	struct rq *rq;
	struct rb_node rb_node;
};
\end{lstlisting}

\begin{itemize}
\item \texttt{oxc\_nr\_running} is the number of oxc tasks enqueued in 
		the container's local runqueue. We say these tasks work 
		in the ox container.
\item \texttt{oxc\_throttled} is set when an ox container runs out of
		its budget in a period.	Here ''throttled'' is a 
		convention inherited from RT throttling and 
		maps to the suspended state of a CBS. 
\item \texttt{oxc\_deadline} is current deadline of this ox container,
		which is a server in CBS theory.
\item \texttt{oxc\_time} is currently consumed budget in a period.
\item \texttt{oxc\_runtime} and \texttt{oxc\_period} are CBS parameters:
		\texttt{oxc\_runtime} is maximum budget and 
		\texttt{oxc\_period} is the period.
\item \texttt{oxc\_period\_timer} is timer which will activate at recharging
		time. If at some point, the value of \texttt{oxc\_time} is 
		larger than the value of \texttt{oxc\_runtime}, then
		the ox runqueue will be throttled and 
		its timer will be set to fire, at the 
		current deadline, \texttt{oxc\_deadline}, 
		to recharge the container.
\item \texttt{oxc\_runtime\_lock} guarantees that the update of an ox 
		container's reservation parameters (\texttt{oxc\_runtime} 
		and \texttt{oxc\_period}) and internal state 
		(\texttt{oxc\_time} and \texttt{oxc\_deadline}) happen
		in a consistant way.
\item \texttt{rq\_} is the local runqueue of the ox container.
\item \texttt{rq} points to a per CPU runqueue and its CPU is where the 
		ox container reserves bandwidth from.
\item \texttt{rb\_node} is used to put an ox runquneue in a red black tree.
		All oxc runqueues reserve bandwidth from the same CPU
		are sorted in a red-black tree. In this tree, an ox 
		container's \texttt{oxc\_deadline} value is used to order 
		nodes.
\end{itemize}
For each CPU, there is a red-black tree which stores all oxc runeueues that
reserve bandwidths in this CPU and orders them with their current deadline.
The oxc runqueue with earliest deadline is stored in the leftmost node. 
This tree is called the edf tree and is defined in 
\texttt{struct oxc\_edf\_tree}.

\begin{lstlisting}[language=C, caption={The EDF tree}]
struct oxc_edf_tree {
	struct rb_root rb_root;
	struct rb_node *rb_leftmost;
};
\end{lstlisting}
The pointer \texttt{rb\_leftmost} helps fast access to the earliest deadline
ox container in a CPU.

An ox container is responsible for reserving bandwidth from a CPU.
Another data structure \texttt{struct hyper\_oxc\_rq} is defined to 
reserve bandwidth from multiple CPUs. A \texttt{struct hyper\_oxc\_rq}
instance is called a hyper ox container. 

\begin{lstlisting}[language=C, caption={The hyper ox container}]
struct hyper_oxc_rq {
	cpumask_var_t cpus_allowed;
	struct oxc_rq ** oxc_rq;
};
\end{lstlisting}
\begin{itemize}
\item \texttt{cpus\_allowed} specifies the CPUs that are used to reserve 
				bandwidth.
\item \texttt{oxc\_rq} is an array of (pointers to) ox containers to 
		reserve bandwidth from CPUs identified in 
		\texttt{cpus\_allowed}.
\end{itemize}

\section{Extensions on original data structures\label{sec:extention}}
Several new data structures have been imported in the kernel that will
involve with Linux schedulers , yet the interfaces defined in 
\texttt{struct sched\_class} are kept unmodified. Extensions are added 
in some original data structures in order to merge newly defined data 
structures in the system. Such extensions are not complex.

\begin{lstlisting}[language=C, caption={Extensions on \texttt{struct rq}}]
struct rq {
	...
	int in_oxc;
	struct oxc_edf_tree oxc_edf_tree;
};
\end{lstlisting}
Two fileds are added in runqueue structure. The \texttt{in\_oxc} is 
used to distinguish per CPU and ox container runqueue.
As the name says, for an ox container's local runqueue, its 
\texttt{in\_oxc} field is set. And \texttt{oxc\_edf\_tree} in a 
per CPU runqueue is the edf tree for a CPU to keep and sort ox
containers.

The kernel configuration option \texttt{CONFIG\_CGROUP\_SCHED} is required
by the oxc framework. This option allows to create arbitrary task groups
using the "cgroup" pseudo filesystem. In current implementation of oxc 
framework, reservation is made for tasks in a control group. Tasks in a 
cgroup is represented by the \texttt{struct task\_group} structure. And 
extensions also happen inside it.

\begin{lstlisting}[language=C, 
			caption={Extensions on 
					\texttt{struct task\_group}}]
struct task_group {
	...
	struct hyper_oxc_rq *hyper_oxc_rq;
	int oxc_label;
};
\end{lstlisting}
Because tasks in a cgroup can span multiple CPUs, \texttt{struct task\_group}
is a good place to put the hyper ox container. If a task group has been
allocated CPU bandwidths through oxc control, then this task group runs 
inside a hyper ox container, and its \texttt{hyper\_oxc\_rq} points 
to that hyper container; otherwise, this field is NULL. As before,
a task group inside a hyper oxc is called an oxc task group.
There are three types of oxc task groups: an oxc group whose father
is not an oxc task group; an oxc task group whose father is an oxc group
with a different hyper ox container; an oxc task group with the same hyper
ox container as its father. The \texttt{oxc\_label} field is used to 
differ them. For non oxc task group, this field is not utilized.

When oxc scheduling is enabled in the kernel, there are two kinds of tasks
in the system: oxc tasks and non oxc tasks. To difference between them
is that oxc tasks work is (or will be) enqueued in an ox container's local 
runqueue and non oxc tasks work with a per CPU runqueue. So, from a task, 
its associated runqueue can be tracked according to the scheduling route 
in figure~\ref{fig:scheduling_route_oxc} or~\ref{fig:oxc_task_no_tg}.
As long as the runqueue is found, given its \texttt{in\_oxc} field, the 
status of this runqueue and the task can both be fixed. Consider that 
the question ''is that an oxc task?'' happens frequently in the framework, 
a \texttt{is\_oxc\_task} field is added in \texttt{struct task\_struct} for 
efficient recognition of an oxc task.

\begin{lstlisting}[language=C, caption={\texttt{is\_oxc\_task} field in 
						\texttt{struct task\_struct}}]
struct task_struct {
	int is_oxc_task;
	...
};
\end{lstlisting}
When a task runs in an ox container, this new field is set.

\section{To direct a task to a per ox container runqueue\label{sec:redir}}

In section~\ref{sec:design_oxc_scheduling}, we show the scheduling routes
when there exists oxc scheduling in a system. This section will introduce
the details on how to build these scheduling routes.

\subsection{To build the scheduling route in mainline Linux}

In order to schedule a task in a per oxc runqueue, the first thing is to 
associate this task with the local runqueue of an oxc. To understand this,
let's first see how the system associates a task to a runqueue in mainline 
Linux, where there is only per CPU runqueues. This is done through
the method \texttt{set\_task\_rq}.

\begin{lstlisting}[language=C, label={lst:set_task_rq},
	caption={To associate tasks with a per CPU runqueue in mainline Linux}]
void set_task_rq(struct task_struct *p, unsigned int cpu)
{
#ifdef CONFIG_FAIR_GROUP_SCHED
        p->se.cfs_rq = task_group(p)->cfs_rq[cpu];
        p->se.parent = task_group(p)->se[cpu];
#endif

#ifdef CONFIG_RT_GROUP_SCHED
        p->rt.rt_rq  = task_group(p)->rt_rq[cpu];
#endif
}
\end{lstlisting}

As demonstrated in list~\ref{lst:set_task_rq}, codes inside 
\texttt{set\_task\_rq} build up the front part of the scheduling route 
when \texttt{CONFIG\_FAIR\_GROUP\_SCHED} and 
\texttt{CONFIG\_RT\_GROUP\_SCHED} are set.
When RT or CFS task group scheduling is enabled, each task is then directed 
to its task group. In mainline Linux, the second part of a scheduling route
only directs a task group to the per CPU runqueues. Such paths are connected 
when the task group is created. Figure~\ref{fig:tg_creation} shows the hint how 
a task group joins the scheduling route during its creation. In addition, now 
we know that a scheduling route is built backwards from an runqueue end.

\begin{figure}[htbp]
        \centering
        \includegraphics[width=\textwidth]{images/tg_creation}
        \caption{The creation of a task group in original Linux}
        \label{fig:tg_creation}
\end{figure}
In case that task group scheduling is not enabled, recall the scheduling path
where the \texttt{task\_rq} leads a task to its per CPU runqueue directly.

\subsection{To build the scheduling route in oxc enabled Linux}

Now, we have seen the point when a task or task group joins the scheduling 
route in mainline Linux. These points are still time to fill elements in 
scheduling routes, figure \ref{fig:scheduling_route_oxc} and 
\ref{fig:oxc_task_no_tg}, after oxc runqueues are imported in Linux.

Previously, the \texttt{set\_task\_rq} method does not deal with the task 
that is not in a RT or CFS task group. This is because for tasks without 
group scheduling, the scheduling route \texttt{task\_rq} is utilized.
However, \texttt{task\_rq} does not work for an oxc task to locate the
per container runqueue. The functionality of \texttt{set\_task\_rq} is
then extended to care about tasks without group scheduling. 

\begin{lstlisting}[language=C, 
        caption={The extended \texttt{set\_task\_rq}}]
void set_task_rq(struct task_struct *p, unsigned int cpu)
{
        struct task_group *tg = task_group(p);

#ifdef CONFIG_FAIR_GROUP_SCHED
        p->se.cfs_rq = tg->cfs_rq[cpu];
        p->se.parent = tg->se[cpu];
#else
        if(!tg->hyper_oxc_rq)
                p->se.cfs_rq = &cpu_rq(cpu)->cfs;
        else
                p->se.cfs_rq = &tg->hyper_oxc_rq->oxc_rq[cpu]->rq_.cfs;
#endif

#ifdef CONFIG_RT_GROUP_SCHED
        p->rt.rt_rq  = tg->rt_rq[cpu];
        p->rt.parent = tg->rt_se[cpu];
#else
        if(!tg->hyper_oxc_rq)
                p->rt.rt_rq = &cpu_rq(cpu)->rt;
        else
                p->rt.rt_rq = &tg->hyper_oxc_rq->oxc_rq[cpu]->rq_.rt;
#endif
        if (rq_of_task(p)->in_oxc == 1)
                p->is_oxc_task = 100;
        else
                p->is_oxc_task = 0;
}
\end{lstlisting}
                                                  
When task group scheduling is enabled, there is no difference for setting a 
task's runqueue in both
the mainline Linux and oxc enabled Linux. The interesting part happens when 
task group scheduling is not set. This time, given that the task group is 
associated with a hyper oxc or not, the task is drected to a per CPU runqueue
or per an oxc runqueue ( the container is supplied by the 
\texttt{hyper\_oxc\_rq}). This corresponds the first part of the scheduling 
route in figure~\ref{fig:oxc_task_no_tg}. In the end of the method, 
\texttt{is\_oxc\_task} label of a task is configured. When 
\texttt{task\_rq\_oxc} is invoked, the scheduling route for a task has already 
been set. It tracks the shceduling route to find the runqueue that a task is 
just associated. After this extended \texttt{set\_task\_rq} call, both routes 
with or without group scheduling are built up.

In the mainline Linux, the end part of a scheduling route is built up when
a task group is created. Within the oxc enabled kernel, things will be a 
little more complex. In the oxc applied Linux, a task group can be associated 
with a hyper oxc and the contained runqueues in three cases: 1) If its parent 
is associated with a hyper oxc, then when it is created it will inherit its 
parent's hyper ox container. 2) The task group is explicitly attached to
a hyper oxc. 3) When the group's one ascendant task group is attached to 
a hyper oxc, its \texttt{hyper\_oxc\_rq} field will point to that hyper 
oxc too. 

\begin{figure}[htbp]
        \centering
        \includegraphics[height=0.25\textheight,width=0.5\textwidth]{images/tg_creation_oxc}
        \caption{The creation of a task group in oxc enabled Linux}
        \label{fig:tg_creation_oxc}
\end{figure}

Corresponding to case 1, now when a task group is created, there 
will be a initilization routine for oxc scheduling. A sketch is shown in 
figure \ref{fig:tg_creation_oxc}. The details of this routinue 
\texttt{alloc\_oxc\_sched\_group} is shown below. 

\begin{lstlisting}[language=C,
        caption={OXC scheduling related initilization 
					during task group creation}]
int alloc_oxc_sched_group(struct task_group *tg, struct task_group *parent)
{
        int i;

        tg->hyper_oxc_rq = parent->hyper_oxc_rq;
        if( parent->hyper_oxc_rq) {
                for_each_possible_cpu(i) {
#ifdef CONFIG_FAIR_GROUP_SCHED
                        tg->cfs_rq[i]->rq =
                                &tg->hyper_oxc_rq->oxc_rq[i]->rq_;
                        if( !parent->se[i] && tg->se[i])
                                tg->se[i]->cfs_rq =
                                      &tg->hyper_oxc_rq->oxc_rq[i]->rq_.cfs;
#endif
#ifdef CONFIG_RT_GROUP_SCHED
                        tg->rt_rq[i]->rq =
                                &tg->hyper_oxc_rq->oxc_rq[i]->rq_;
                        if( !parent->rt_se[i] && tg->rt_se[i])
                                tg->rt_se[i]->rt_rq =
                                       &tg->hyper_oxc_rq->oxc_rq[i]->rq_.rt;
#endif
                }
                tg->oxc_label = 100;
        }
        else
                tg->oxc_label = 0;

        return 1;
}
\end{lstlisting}

The \texttt{alloc\_oxc\_sched\_group} handles oxc related initialization when a 
new task group is created. At first, the a newly created task group will inherit
its parent task group's hyper container. If the parent is an oxc task group, 
this child task group will be directed to per oxc runqueues contained in the
hyper oxc; the result corresponds to the end part of a scheduling route.  
And the \texttt{oxc\_label} for such a child oxc task group is 100.

Case 2 and 3 actually happen at the same time. When explicitly putting the 
parent group inside a hyper ox container, the dscendant task groups will also 
be directed to this hyper oxc. Two methods \texttt{init\_tg\_cfs\_entry\_oxc} and
\texttt{init\_tg\_rt\_entry\_oxc} will be involved in this procedure to 
associate task groups with the hyper ox container. 
The structure in the two is silimar. As an example, Here we only study how the 
CFS part of a task group is handled through \texttt{init\_tg\_cfs\_entry\_oxc}.

\begin{lstlisting}[language=C, caption={To explicitly direct a task group 
						(CFS part) to an OXC \\
					\indent\hspace{5cm} local runqueue}]
static void init_tg_cfs_entry_oxc(struct task_group *tg,
					struct cfs_rq *cfs_rq,
					struct sched_entity *se, int cpu,
					struct sched_entity *parent,
					struct oxc_rq *oxc_rq)
{
	struct rq *rq = rq_of_oxc_rq(oxc_rq);
	init_tg_cfs_entry(tg, cfs_rq, se, cpu, parent);
	tg->cfs_rq[cpu]->rq = rq;
	if( !parent && se)
		se->cfs_rq = &rq->cfs;
} 
\end{lstlisting}
A brief explanation on the parameters:
\begin{itemize}
\item \texttt{tg} is the task group to be dealt with.
\item \texttt{cfs\_rq} is the CFS runqueue that this task group 
		owns. 
\item \texttt{se} is the CFS entity that represents \texttt{tg}.
\item \texttt{cpu} specifies the cfs\_rq pointer inside \texttt{tg} that
		will be redirected. The function 
		\texttt{init\_tg\_cfs\_entry\_oxc}
		redirects one CFS runqueue inside a \texttt{tg} to an oxc 
		runqueue, \texttt{oxc\_rq}, in each call. A hyper ox 
		container can have more than one oxc runqueues inside
		and \texttt{tg} has multiple CFS runqueues too.
		To relate a task group with a hyper ox container requires 
		the \texttt{init\_tg\_cfs\_entry\_oxc} be called multiple times.
\item \texttt{parent} points to the parent CFS scheduling entity.
\item \texttt{oxc\_rq} contains the destinated runqueue. 
\end{itemize}

As for codes, first \texttt{rq\_of\_oxc\_rq} returns the oxc local runqueue.
The mainline kernel method \texttt{init\_tg\_cfs\_entry} is then called to 
re-initialize CFS related work for \texttt{tg}. 
%This is a mainline Linux kernel function called in
%task group creation if CFS group scheduling is set.
The real bridging point is in the following line where 
\texttt{tg} connects its local \texttt{cfs\_rq} on 
\texttt{cpu} to the per oxc runqueue just got. After 
\texttt{init\_tg\_cfs\_entry\_oxc} is invoked over every CPU for the 
task group, CFS tasks and task groups under \texttt{tg} will work in 
the scheduling route leading them to an oxc local runqueue. When a task 
group is explicitly directed to a hyper ox container, the whole family of 
task groups under it will also be associated with this hyper ox container. 
This task group will be the top of this hierarchy, and it will be enqueued 
in the per oxc runqueue's embedded CFS runqueue directly. The last 
\texttt{if} condition returns true when \texttt{tg} is top in the oxc 
group hierarchy.
%One amazing feature of this procedure to relate a task group to a per oxc 
%runqueue is that it has nothing to do tasks under this group.

\section{Run tasks under OXC scheduling framework\label{sec:run}}

As long as per oxc runqueue joins the scheduling route, modular schedulers
can handle tasks in an ox container without differentiating oxc tasks from
non oxc tasks.
%the scheduling 
%of tasks is compatible with modular schedulers in Linux. 
In case to schedule an oxc task, just pass the task itself and its oxc 
local runqueue, instead of per CPU runqueue, to its corresponding 
scheduling class. The local runqueue of an oxc task can be easily tracked 
along the scheduling route. And the scheduler will behave as usual. That 
is, this oxc scheduling framework is transparent to both schedulers and 
tasks.

Because we consider reserving CPU bandwidth for an ox container, there
are scheduling operations that before or after passing parameters to them,
the reservation information should be updated. For these kinds of scheduling
operations, we adapt a relaying mechanism. The parameter is first passed to
another function and after necessary actions, the scheduling operation is
called inside this function. We say that such scheduling operations are
encapsulated in oxc (scheduling) functions.
Still scheduling details for a task are not the framework's work. 

In order to fulfil real time guarantee, oxc tasks are always privileged
to non oxc tasks. Among oxc tasks inside a container, the priority relation
is the same as in Linux. The ox container itself can be considered as a 
virtual Linux system.

For each scheduling operation defined in \texttt{struct sched\_class},
there are three fates for them under oxc scheduling framework: 
some are encapsulated inside oxc functions so as to work with oxc tasks; 
some can work under the framework directly; and others are not supported. 
The table~\ref{tab:op_classes} displays the three classes of scheduling 
operations. The naming convention for oxc functions which encapsulate 
a scheduling operation inside is appending the original name with 
\texttt{\_oxc}. For example, the scheduling operation
\texttt{task\_tick} is called inside in \texttt{task\_tick\_oxc}.
The enqueue and dequeue are two exceptions, they are enclosed in 
\texttt{enqueue\_task\_oxc} and \texttt{dequeue\_task\_oxc}.

\begin{table}[thbp]
  \centering
  \begin{tabular}{l|l|l}\hline
	\small{Work inside an oxc function} & \small{Work without encapsulation} & \small{Unsupported}\\\hline
		check\_preempt\_curr	& yield\_task				& Others	\\
		pick\_next\_task	& yield\_to\_task			&		\\
		put\_prev\_task		& task\_waking				&		\\
		set\_curr\_task		& task\_woken				&		\\
		task\_tick		& set\_cpus\_allowed			&		\\
		enqueue\_task\_rq	& task\_fork				&		\\
		dequeue\_task\_rq	& switched\_from			&		\\
					& switched\_to				&		\\
					& prio\_changed				&		\\
					& get\_rt\_interval			&		\\
					& task\_move\_group			&		\\\hline
  \end{tabular}	
  \caption{The way to handle a scheduling operation under the oxc framework}
  \label{tab:op_classes}
\end{table}

The following will be descriptions on oxc functions with emphasis on ones that
encapsulate a scheduling operation inside.

\subsection{To obtain the runqueue of a task\label{sec:rq_of_task}}
A method \texttt{rq\_of\_task}, listing~\ref{lst:rq_of_task}, is used to 
obtain the runqueue of a task. The runqueue retuned can be an oxc local 
runqueue or per CPU one depending that whether the task is inside a 
container. This function can be used to replace the \texttt{task\_rq} 
macro under oxc framework. For any task, it has both CFS scheduling 
entity and RT entity. And the RT and CFS scheduling routes both exists 
for a task. Here, we exploit the CFS scheduling route. Given 
\texttt{CONFIG\_FAIR\_GROUP\_SCHED} is set or not, the corresponding 
scheduling route is explored to track the runqueue.

\begin{lstlisting}[language=C, 
		caption={Codes to obtain the runqueue of a task},
			label={lst:rq_of_task}]
struct rq* rq_of_task(struct task_struct *p)
{
	struct rq *rq;

#ifdef CONFIG_FAIR_GROUP_SCHED
	rq = p->se.cfs_rq->rq;
#else
	rq = task_rq_fair_oxc(p);
#endif
	return rq;
}
\end{lstlisting}

One important use of this function is in \texttt{oxc\_rq\_of\_task}, 
which takes a task as the input, and returns the ox container the 
task is inside. In case the input is not an oxc task, 
\texttt{NULL} will be returned. The \texttt{oxc\_rq\_of\_task} is 
one most often invoked method under oxc framework. It first gets a 
task's runqueue using the \texttt{rq\_of\_task}, then obtains the ox 
container the runqueue is in, in case the runqueue is in an container.

\subsection{To enqueue an oxc task\label{sec:enqueue_task_oxc}}

When an oxc task arrives, besides enqueue it in the oxc local runqueue,
the ox container's reservation information may be updated if necesary.
The \texttt{enqueue\_task\_oxc} shows the typical scheme for oxc 
functions to enclose a scheduling operation.

\begin{lstlisting}[language=C, 
		caption={To enqueue an task into the oxc local runqueue}]
void enqueue_task_oxc(struct rq *rq, struct task_struct *p, int flags)
{
	struct oxc_rq *oxc_rq = oxc_rq_of_task(p);
	struct rq *rq_ = rq_of_oxc_rq(oxc_rq);

	/* Update the local runqueue' clock. */
	update_rq_clock(rq_);

	/*	
	* Enqueue the task into the local runqueue
	* by its scheduling class.
	*/
	p->sched_class->enqueue_task(rq_, p, flags);

	inc_oxc_tasks(p, oxc_rq);
	enqueue_oxc_rq(oxc_rq);
}
\end{lstlisting}

The method \texttt{oxc\_rq\_of\_task} tracks the scheduling route and 
returns the ox runqueue of an oxc task. Then the ox container's local 
runqueue's time information is updated. Although the ox container does 
not care about scheduling details of tasks inside it, tasks are indeed 
enqueued in its local runqueue and may rely the runqueue's time information. 
We can see that all scheduling details are carried out by a task's 
scheduling class as the \texttt{enqueue\_task} operation of the 
scheduling class is called with the task and local runqueue as inputs. 
The \texttt{inc\_oxc\_tasks} method simply increases the number of oxc 
tasks in the oxc runqueue by one. The reason that such a simple function 
is stiil kept is that as the framework grows more complex in the future, 
extra operations can be put in this method.

\begin{lstlisting}[language=C, 
		caption={To update the number of tasks inside a container} ]
static inline void 
inc_oxc_tasks(struct task_struct *p, struct oxc_rq *oxc_rq)
{
        oxc_rq->oxc_nr_running ++;
}
\end{lstlisting}

Until now, the oxc task has been put in the local runqueue. 
If before the arrival of this task the ox container is empty and not 
throttled, it is time to put this container in its EDF tree.
This is the work of \texttt{enqueue\_oxc\_rq} method. 

\begin{lstlisting}[language=C, 
		caption={To put an ox container in the EDF tree
					if necessary}]
static void enqueue_oxc_rq(struct oxc_rq *oxc_rq)
{
        int on_rq;

        on_rq = oxc_rq_on_rq(oxc_rq);

        BUG_ON(!oxc_rq->oxc_nr_running);
        BUG_ON(on_rq && oxc_rq_throttled(oxc_rq));

        if( on_rq) {
                /* Already queued properly. */
                return;
        }
        /* We do not put a throttled oxc_rq in the edf tree. */
        if( oxc_rq_throttled(oxc_rq))
                return;

        oxc_rq_update_deadline(oxc_rq);
        __enqueue_oxc_rq(oxc_rq);
}
\end{lstlisting}

\texttt{on\_rq} tells if the container \texttt{oxc\_rq} is in a edf tree.
\texttt{BUG\_ON} is a Linux kernel macro. If the condition it checks is 
\texttt{true}, then the kernel will crash! Because we just put a task
in the local runqueue, so the first \texttt{BUG\_ON} should be passed.
When an ox container runs out of its budget, it should be moved from the
edf tree, this is what the second \texttt{BUG\_ON} checks. Now we pass
the two \texttt{BUG\_ONs}. If the \texttt{oxc\_rq} is already on edf 
tree or moved from the tree because of exhausting budget, nothing needs 
to be done. There are two conditions for an ox container to be outside
an edf tree: it is throttled or it is empty. If the last \texttt{if}
condition is passed, this means a task just joins an empty container. 
Recall the CBS rules ''when a task arrives and the server is idle, update 
the deadline if necessary''. This is exactly what 
\texttt{oxc\_rq\_update\_deadline} does. When it comes to CPU reservation,
an ox container corresponds to a constant bandwidth server in CBS theory.
\texttt{\_enqueue\_oxc\_rq} is a quite mechanical procesure to put an
oxc runqueue in an edf tree.

\subsection{To dequeue an oxc task\label{sec:dequeue_task_oxc}}

\texttt{dequeue\_task\_oxc} is the opposite method of 
\texttt{enqueue\_oxc\_rq}.

\begin{lstlisting}[language=C, 
		caption={To remove a task from the oxc local runqueue}]
static void 
dequeue_task_oxc(struct rq *rq, struct task_struct*p, int flags)
{
        struct rq *rq_ = rq_of_task(p);
        struct oxc_rq *oxc_rq = container_of(rq_, struct oxc_rq, rq_);

        /* Update the local runqueue. */
        update_rq_clock(rq_);
        /*
         * Dequeue the task from the local runqueue 
         * by its scheduling class.
         */
        p->sched_class->dequeue_task(rq_, p, flags);

        dec_oxc_tasks(p, oxc_rq);
        dequeue_oxc_rq(oxc_rq);
}
\end{lstlisting}

The layout inside \texttt{dequeue\_task\_oxc} is the same as what we see in
\texttt{enqueue\_task\_oxc}: local runqueue's clock information is updated,
local runqueue and task are relayed to the corresponding scheduler, 
the task number and EDF tree are updated. When an oxc task leaves the 
container, it is the time to check if the oxc is empty or not, which is 
done in \texttt{dequeue\_oxc\_rq}.

\begin{lstlisting}[language=C,
		caption={To remove an ox container 
				from the EDF tree if necessary}]
static void dequeue_oxc_rq(struct oxc_rq *oxc_rq)
{
        int on_rq;

        on_rq = oxc_rq_on_rq(oxc_rq);
        /*
         * Here we do not expect throttled oxc_rq to be in the 
         * edf tree. Note that when an oxc_rq exceeds its 
         * maximum budget, it is dequeued via 
         * sched_oxc_rq_dequeue().
         */
        BUG_ON(on_rq && oxc_rq_throttled(oxc_rq));
        /* 
         * If an oxc_rq is not in the edf tree, it should 
         * be throttled or have no tasks enqueued.
         */
        BUG_ON(!on_rq && !oxc_rq_throttled(oxc_rq) && !oxc_rq->oxc_nr_running);

        if( on_rq && !oxc_rq->oxc_nr_running) {
                /* Dequeue the oxc_rq if it has no tasks. */
                __dequeue_oxc_rq(oxc_rq);
                return;
        }
}
\end{lstlisting}

The comments are explanable enough. \texttt{\_\_dequeue\_oxc\_rq} 
is the counterpart of \texttt{\_\_enqueue\_oxc\_rq} to remove an oxc
runqueue from the EDF tree.

\subsection{To check the preemption condition
			\label{sec:check_preempt_curr_oxc}}
When a task wakes up from sleeping or is created, the scheduler will check 
if it can preempt temporarily running task in the same CPU. If the current 
task is an oxc task and the waking task is not an oxc task, the later 
cannot preempt the current one. If both are non oxc tasks, Linux already 
has methods to check. So here we only pay attention to the situation
when the waking task is an oxc task.

\begin{lstlisting}[language=C, 
		caption={The preemption check for an oxc task},
		label={lst:check_preempt_curr_oxc}]
static inline int
check_preempt_oxc_rq(struct task_struct *curr, struct task_struct *p, int flags)
{
        struct oxc_rq *oxc_rq = oxc_rq_of_task(p);
        struct oxc_rq *oxc_rq_curr = oxc_rq_of_task(curr);
        const struct sched_class *class;

        if(oxc_rq_throttled(oxc_rq)
              return 0;
        /* 
         * Tasks from a unthrottled oxc_rq always has a higher 
         * priority than non oxc tasks.
         */
        if( !oxc_rq_curr)
                return 1;

        /* Both p and current task are in the same oxc_rq. */
        if( oxc_rq_curr == oxc_rq) {
                if( p->sched_class == curr->sched_class) 
                        curr->sched_class->check_preempt_curr(
                                      &oxc_rq->rq_, p, flags);
                else {
                        for_each_class(class) {
                                if( class == curr->sched_class)
                                        break;
                                if( class == p->sched_class) {
                                        resched_task(curr);
                                        break;
                                }
                        }
                }

                return 0;
        }

        /* 
         * p and current tasks are oxc tasks from 
         * different ox containers. 
         */
        return oxc_rq_before(oxc_rq, oxc_rq_curr);
}
\end{lstlisting}

\texttt{curr} and \texttt{p} are the current task and waking task 
respectively. If task \texttt{p}'s container is throttled, it cannot 
preempt currently running task. Otherwise, if \texttt{curr} is not an 
oxc task, the privilege is given to oxc task \texttt{p}.
When both \texttt{p} and \texttt{curr} are oxc tasks and are contained 
in the same oxc runqueue: if they are even in the same scheduling class, 
it's the modular scheduler's responsibility to decide if a preemption 
can happen or not; otherwise, the task whose scheduling class has higher 
priority in the scheduler chain is chosen to run. In the last case, they 
two are from different ox containers. Now, \texttt{oxc\_rq\_before} 
checks if a contaienr's deadline is before another's deadline. The one 
with earlier deadline will run. 

\subsection{To pick up an oxc task\label{sec:pick_next_task_oxc}}

When to pick a most eligible task to run in a system, oxc tasks should 
be checked first. If there is no eligible oxc tasks, then non oxc tasks 
are considered. \texttt{pick\_next\_task\_oxc} is responsible for choosing 
the most eligible oxc task in a CPU.

\begin{lstlisting}[language=C, label={lst:pick_next_task_oxc},
		caption={To pick up the most eligible oxc task}]
static struct task_struct* pick_next_task_oxc(struct rq *rq)
{
        struct oxc_rq *oxc_rq;
        struct rq *rq_;
        struct task_struct *p, *curr;
        const struct sched_class *class;
        /* This clock update is necessary! */
        update_rq_clock(rq);
        update_curr_oxc_rq(rq);
        oxc_rq = pick_next_oxc_rq(rq);
        if( !oxc_rq)
                return NULL;

        rq_ = rq_of_oxc_rq(oxc_rq);

        update_rq_clock(rq_);

        for_each_class(class) {
                if( class != &idle_sched_class) {
                        p = class->pick_next_task(rq_);
                        if( p) {
                                rq_->curr = p;
                                return p;
                        }
                }
        }

        return NULL;
}
\end{lstlisting}

Inside the oxc function \texttt{pick\_next\_task\_oxc},
there is one thing to note: not only local runqueue's clock
is updated, but also the per CPU runqueue's clock is updated here.
This is because the reservation time of an ox container is counted
using the per CPU runqueue's clock and to keep the clock on time
for the container's use, here it is updated. 
\texttt{pick\_next\_oxc\_rq} is called to pick the ox runqueue with the 
earliest deadline in a CPU. Then along the scheduling class chain, each
scheduler uses its scheduling operation trying to find the most eligible
task in the ox container's local runqueue. A very important method here 
is \texttt{update\_curr\_oxc\_rq}. The budget comsumption of an ox 
container actually happens in this method.

\begin{lstlisting}[language=C, 
		caption={To update an ox container's runtime information}]
static void update_curr_oxc_rq(struct rq *rq)
{
        struct task_struct *curr = rq->curr;
        struct oxc_rq *oxc_rq = oxc_rq_of_task(curr);
        u64 delta_exec;
        /*
         * If current task is not oxc task, simply return.
         */
        if( !oxc_rq)
                return;

        delta_exec = rq->clock - oxc_rq->oxc_start_time;
        oxc_rq->oxc_start_time = rq->clock;
        if( unlikely((s64)delta_exec < 0))
                delta_exec = 0;

        raw_spin_lock(&oxc_rq->oxc_runtime_lock);

        oxc_rq->oxc_time += delta_exec;
        if( sched_oxc_rq_runtime_exceeded(oxc_rq)) {
                resched_task(curr);
        }

        raw_spin_unlock(&oxc_rq->oxc_runtime_lock);
}
\end{lstlisting}

\texttt{update\_curr\_oxc\_rq} updates the runtime value of 
an oxc runqueue. If the budget in current period is exhausted, the
current task needs to be rescheduled. the local spinlock 
\texttt{oxc\_runtime\_lock} protects the update of runtime from
interleave. The \texttt{sched\_oxc\_rq\_runtime\_exceeded} is
used to check if the ox container has exceeded its budget in a
period.

\begin{lstlisting}[language=C, 
	caption={To check if an ox container should be throttled}]
static int sched_oxc_rq_runtime_exceeded(struct oxc_rq *oxc_rq)
{
        u64 runtime = sched_oxc_rq_runtime(oxc_rq);
        u64 period = sched_oxc_rq_period(oxc_rq);

        /* 
         * If the runtime is set as 'RUNTIME_INF',
         * the ox container can run without throttling.
         */
        if( runtime == RUNTIME_INF)
                return 0;

        /* 
         * If the runtime to be larger the the period,
         * the ox container can run without throttling.
         */
        if( runtime >=period)
                return 0;

        /* There is still budget left. */
        if( oxc_rq->oxc_time < runtime)
                return 0;
        /* 
         * The reservation in a period has been exhausted,
         * to set the throttling label, remove the oxc_rq
         * from the edf tree and start the recharging timer.
         */
        else {
                oxc_rq->oxc_throttled = 1;
                sched_oxc_rq_dequeue(oxc_rq);
                start_oxc_period_timer(oxc_rq);

                return 1;
        }
}
\end{lstlisting}

Inside \texttt{sched\_oxc\_rq\_runtime\_exceeded}, at first
a series of non exceeded conditions is checked, which is 
easy to understand. The last \emph{else} statement deals with
the case that the container should be throttled: the 
\texttt{oxc\_throttled} label is set, the oxc runqueue is removed 
from the edf tree and the timer is set to fire at the current 
deadline, \texttt{oxc\_deadline}, to recharge the budget.

\subsection{put\_prev\_task\_oxc\label{sec:put_prev_task}}

The scheduling operation \texttt{put\_prev\_task} is called when 
the currently running task is possible to be replaced. It performs some 
conclusion work for the task. Although the currently running task may 
keep running without giving up the CPU. If currently running task is an 
oxc task, when \texttt{put\_prev\_task} is called, this is also a point 
to update the ox container runtime information. 

\begin{lstlisting}[language=C, label={lst:put_prev_task_oxc},
		caption={Conclusion work before an oxc task is switched 
				out of a CPU}]
static void 
put_prev_task_oxc(struct rq* rq, struct task_struct *p)
{
        struct rq *rq_ = rq_of_task(p);

        update_rq_clock(rq_);
        update_curr_oxc_rq(rq);

        p->sched_class->put_prev_task(rq_, p);
}
\end{lstlisting}

Now, when a \texttt{put\_prev\_task} operation is needed and the
currently running task is an oxc task, instead of calling the 
\texttt{put\_prev\_task} defined in a scheduling class directly, our
\texttt{put\_prev\_task\_oxc} encapsulation will be called first then
ox container local runqueue and current task will be relayed to
the corresponding scheduler.

\subsection{set\_curr\_task\_oxc\label{sec:set_curr_task_oxc}}
%The scheduling operation \texttt{put\_prev\_task} is the last scheduling 
%operation before a task gives up a CPU (of course, if it is chosen again 
%immediately, it can still occupy the CPU). 
The \texttt{put\_prev\_task}/\texttt{set\_curr\_task} is used whenever
the current task is changing policies or groups. Similiar as what happens 
to \texttt{put\_prev\_task}, there is an oxc function 
\texttt{set\_curr\_task\_oxc} that performs the operation 
\texttt{set\_curr\_task\_oxc} inside and start an ox container's
budget counting.

\begin{lstlisting}[language=C,
	caption={Another point to update reservation information},
	label={lst:set_curr_task_oxc}]
static void set_curr_task_oxc(struct rq *rq)
{
        struct task_struct *curr = rq->curr;
        struct rq *rq_ = rq_of_task(curr);
        struct oxc_rq *oxc_rq;

        oxc_rq = container_of(rq_, struct oxc_rq, rq_);

        oxc_rq->oxc_start_time = oxc_rq->rq->clock;

        update_rq_clock(rq_);
        curr->sched_class->set_curr_task(rq);
}
\end{lstlisting}

One thing to note in listing~\ref{lst:set_curr_task_oxc} is that 
this time the per CPU runqueue parameter is passed to the 
task's corresponding scheduling class. First of all, this is feasible 
because \texttt{set\_curr\_task} operation updates the infomation in the 
scheduling route excluding the runqueue (at least in scheduling classes
in mainline Linux) and the current task in the per CPU runqueue
and current task in the earliest dealine oxc runqueue are the same one. 
So, there is no difference to pass which runqueue. 
The real reason is that there is a possible insconsistent state under 
current oxc framework. Initially, the current task in an ox 
container's local runqueue is 
NULL\footnote{This will be improved in future work}, which is different
from the per CPU runqueue's behaviour that the default current task would 
be the special idle task from scheduling class \texttt{sched\_idle}. 
At this time, if the current task is moved 
into this ox container, the \texttt{set\_curr\_task} operation is called 
even before the task is really enqueued into the container's local runqueue. 
That is, the current task in the local runqueue is still \texttt{NULL} when
\texttt{set\_curr\_task} is called and this will cause problems. Thus, a 
per CPU runqueue is used here temporarily.

\subsection{task\_tick\_oxc\label{sec:task_tick_oxc}}

The \texttt{task\_tick} operation is the most frequently called 
scheduling operation and is used to update the task's timing 
information. 
The oxc control also utilizes this opportunity to update the 
current ox container's runtime information and 
local runqueue when the CPU is occupied by an oxc task.

\begin{lstlisting}[language=C, label={lst:task_tick_oxc},
		caption={The most frequently called entry to update a container's runtime}]
			
static void 
task_tick_oxc(struct rq *rq, struct task_struct *p, int queued)
{
        struct rq *rq_ = rq_of_task(p);

        update_curr_oxc_rq(rq);
        update_rq_clock(rq_);

        p->sched_class->task_tick(rq_, p, queued);
}
\end{lstlisting}

\section{SMP support in oxc framework}
An ox container based scheduling system acts with no difference as the 
per CPU scheduling system. Inside different ox containers, the 
scheduling is performed independently and a group of ox containers
can be arranged in one hyper container to realize multi-processor 
bandwidth reservation. For a hyper ox container, tasks are partitioned 
to each ox container and task migration or load balancing between 
different ox contaienrs in a hyper container are missing now.

\section{User interfaces provided by OXC framework}
Currently, the user interfaces, mainly for test purposes, 
of oxc framework are based on cgroup virtual
filesystem. The CPU reservation functionality is realized through 
\emph{cpu} cgroup subsystem. This \emph{cpu} cgroup subsystem is for CPU 
bandwidth control in Linux. RT throttling, CFS group scheduling and 
CFS bandwidth control are all realized through it. The following demonstrates 
how to use the oxc framework. The hardware platform is assumed to have dual
processors.\\
To mount the \emph{cpu} cgroup subsystem(in directory /cgroup):
\begin{lstlisting}
	#mount -t cgroup -ocpu none /cgroup
\end{lstlisting} 
To create a cgroup for CPU reservation :
\begin{lstlisting}
	#mkdir -p /cgroup/cg
\end{lstlisting}
Observe the files inside \texttt{/cgroup/cg} directory, there is one
new file named \texttt{cpu.oxc\_control} and it is the interface to control
CPU bandwidth in oxc framework. However, if one tries to see the content
of this file:
\begin{lstlisting}
	#cat /cgroup/cg/cpu.oxc_control
\end{lstlisting}
It will display nothing. This is because by default the oxc reservation
is disabled until the reservation is triggered for the first 
time. The reservation is triggerred by setting reservation parameters.
To reserve bandwidth in a CPU, three parameters should be specified:
\texttt{CPU number, maximum budget and period}. For example:
\begin{lstlisting}
	#echo 0 100000/1000000 1 20000/500000 > cg/cpu.oxc_control
\end{lstlisting}
This command reserve 100ms every 1s on CPU 0 and 20ms every 500ms on CPU 1
to cgroup \texttt{cg}. 0 and 1 are CPU numbers. 100000 and 20000 are maximum
budgets and 1000000 and 500000 are periods. 
In fact, behind this command, a hyper ox container with two ox containers
inside is created. 
Tasks and task groups inside cgroup \texttt{cg} and its
descedant will be work inside these two ox containers since now.
The two containers' reservation parameters are specified
by the command inputs. 
The unit for budget and period value in the command 
is micro second. This follows the convention in \emph{cpu} cgroup
subsystem, since when you use other bandwidth control mechanisms the value 
is also considered in micro seconds. Tasks inside this cgroup and its further 
sub cgroups will run using the above reserved CPU bandwidth. Now we can say 
\texttt{cg} is contained in a hyper ox container.

Now to \texttt{cat} the content of the \texttt{cpu.oxc\_control} interface 
under directory \texttt{/cgroup/cg}:
\begin{lstlisting}
	#cat /cgroup/cg/cpu.oxc_control
\end{lstlisting}
The reservation parameters we just set will be displayed:
\begin{lstlisting}
	0 100000/1000000 1 20000/500000
\end{lstlisting}
So the file \texttt{oxc\_control} is an interface used for both setting up
and displaying reservation parameters. Reservation parameters can be 
configured in the same way as they are initially set. Furthermore,
there is no need to set up or configure reservation parameters for two 
CPUs at the same time. Suppose in some point, users decide to decrease 
the reservation from CPU 1, they can simply use the following command to
operate on CPU 1 onlu.
\begin{lstlisting}
	#echo 1 20000/1000000 > cg/cpu.oxc_control
\end{lstlisting}
The reservation on CPU 1 is dereased to 20ms every 1s and the reservation
on another CPU is not interfered. 

To create a sub cgroup for cgroup \texttt{cg}:
\begin{lstlisting}
	#mkdir -p /cgroup/cg/cg_0
\end{lstlisting}
The cgroup \texttt{cg\_0} is contained in the same hyper container as
its parent. Try to \texttt{cat} the \texttt{cpu.oxc\_control} file in 
this sub cgroup: 
\begin{lstlisting}
	#cat /cgroup/cg/cg_0/cpu.oxc_control
\end{lstlisting}
An error message will be returned. This is because for a cgroup family
contained in a hyper container, only the top cgroup is allowed for people
to browse and modify reservation parameters.

People can move tasks to cgroup \texttt{cg} and \texttt{cg\_0}. For example
\begin{lstlisting}
	#echo 1982 > cg/tasks
	#echo 1983 > cg_0/tasks
\end{lstlisting}
This moves task with \texttt{pid} 1982 and 1983 to cgroup \texttt{cg} and
\texttt{cg\_0} respectively. Tasks can be RT tasks or normal tasks. All 
tasks inside an ox container behave as working on a virtual Linux system
and utilize the reserved bandwidth.
% Note here we use ''ox container'', not
%''hyper ox container'', because in temporary iplementation, tasks inside
%a hyper ox container are partitioned into each ox containers. 

The oxc tasks can move between different cgroups contained in the same 
hyper container. They can move between ox containers and hyper containers.
They can also leave an ox container and return to be a non oxc task.

Until now, how to reserve CPU bandwidth under oxc framework is introduced.
Now let's see how to distribute the reserved CPU power.
ALthough users cannot browse reservation parameters in cgroup
\texttt{cg\_0}, they can indeed set reservation parameters for 
\texttt{cg\_0}, which will trigger reserved power redistribution.
\begin{lstlisting}
	#echo 0 100000/2000000 1 20000/1000000 > cg/cpu.oxc_control
\end{lstlisting}
After this, a new hyper container with reservation parameters
1000000/1500000 on CPU 0 and 20000/1000000 on CPU 1 will be created
and \texttt{cg\_0} and its descendant cgroups will be associated with
it. Now, although \texttt{cg} and \texttt{cg\_0} are still in the same
hierarchy through the cgroup directory observation, they are indeeded 
contained in two different hyper containers.
The ideal semantics of the above command should also include that the 
reserved bandwidth by \texttt{cg} need to be decreased the same vaue as 
distributed to \texttt{tg\_0}. This behaviour is still missing in current 
prorotype implementation of oxc framework. Yet this is indeed implementable.
Also, the total reserved bandwidth in the system should not be more one in
each CPU; this condition test is not realized either.

\section{Cooperation with scheduling mechanisms inside Linux}

First let's discuss a simple case is when both
\texttt{CONFIG\_FAIR\_GROUP\_SCHED} and \texttt{CONFIG\_RT\_GROUP\_SCHED} 
are not set; that is, task grouping is not 
enabled. In this case, oxc tasks share the reserved CPU bandwidth directly
according to their scheduling policies. In fact, in this situation, the oxc 
control can be used to repeat RT throttling and CFS bandwidth control. 
However, at this time, the reservation is in a real time way and is more 
flexible since CPU bandwidth from different CPUs can be different under 
oxc framework. The result of IRMOS real time scheduler can also be achieved 
by our framework. Consider future scheduling classes that are possible to 
be merged in Linux kernel. For example the \texttt{schedule\_deadline} 
patch, which does not have task grouping scheme by itself. 
Not only that it can work with oxc scheduling naturally becaue of the
ox container's ''open-to-entension'' feature, but also the oxc framework 
can utilize cgroup virtual filesystem as a way to group deadline tasks. 

When the kernel compilation option \texttt{CONFIG\_FAIR\_GROUP\_SCHED} is set, 
the CFS task group scheduling is enabled. Task groups inside the same hyper ox 
container follow the rules of CFS grouping and share the reserved CPU power. 
CFS group scheduling is applied in different areas independently: 
each hyper ox container is an area and outside all ox containers there
is an area. 
%Inside one hyper ox container, bandwidths are reserved
%and task groups inside this hyper ox container share the reserved 
%computation power according to the fairness task group scheduling rules. 

When \texttt{CONFIG\_RT\_GROUP\_SCHED} is set, RT throttling is enabled.
It's necessary to first analyze the possible result when RT throttling 
is applied in a hyper ox container. Suppose inside an container, $Q/T$ 
is the bandwidth reserved from the CPU and RT throttling sets parameters as 
$Q^{'}/T^{'}$ and $Q^{'}/T^{'} \le Q/T$. The ideal behaviour for such an
ox container would be : the container distributes reserved CPU power 
to tasks inside it; and RT tasks inside a container will be throttled when
$Q^{'}$ units of CPU cycles are exhausted in a period $T^{'}$, then non
RT tasks in this container can run. However, in real world a coarse 
combination could cause unreliable results . For example, 
$Q^{'}/T^{'} = 1/10$ and $Q/T = 10/50$. It can happen that there are other 
higher priority containers in the same CPU and in one period the example 
container get the right to use the CPU on the last 10 units of CPU cycles 
in its period. RT tasks inside this container immediately take over the CPU; 
yet after one unit, they are throttled. So, RT tasks only run 1 unit over 
50 unit of CPU cycles although 10 units are reserved. The RT throttling 
result inside a hyper ox container is not stable. Even if we set the period 
parameter in RT throttling the same as its container's, because that RT
throttling and oxc control use different timers, there is a unsynchronization 
between them and can lead to still complex situations.

In a statement, to apply RT throttling naively inside an ox container 
is not efficient. One possible solution is to count the time 
consumption in RT throttling using the ox container's timer, 
this basically means to implement a copy of RT throttling in oxc 
framework itself. In such a case, if some constraints are put, like
the RT throttling period should equal to the container's period, 
we can expect predictable behaviour. Another solution is simply disable 
RT throttling inside an ox container, because oxc scheduling framework 
itself can perform the same result in a real time way as RT throttling. 
In current implementation, there is no communication between RT throttling
and oxc control. And when \texttt{CONFIG\_RT\_GROUP\_SCHED} is set, the 
result is not satisfiable.

The test between CFS bandwidth and oxc control is not conducted yet.
If \texttt{CONFIG\_CFS\_BANDWIDTH} is set, we expect that the situation 
could be similiar to what we see in RT throttling case.

Among cgroup subsystems, there is one \texttt{cpuset} subsystem also 
affecting scheduling behaviour in a cgroup.
After \texttt{cpuset} cgroup subsystem is mounted in a cgroup, there is an 
interface \texttt{cpuset.cpus} appearing in the dorectory. This file 
can control which CPU the tasks inside this cgroup can use.
For example, 
\begin{lstlisting}
	#echo 1 > cpuset.cpus
\end{lstlisting}
This will result tasks inside this cgroup can only run on CPU 1.
In oxc scheduling framework, we have the concept of hyper ox container, which 
control the CPUs that tasks inside it can run in. So, this idea is
compatible with \texttt{cpuset} cgroup subsystem. However, until now the two
mechanisms work independently; future work to bridge the two will make 
the system more consistent.

%%% Local Variables: 
%%% mode: latex
%%% TeX-master: "main"
%%% End: 

\chapter{Experiments\label{chap:exp}}

The overhead introduced by oxc framework includes three parts:
\begin{itemize}
\item The time required to execute codes brought oxc functions.
\item The context switches introduced by the oxc framework.
\item The degradation of modular schedulers' performance under oxc 
	framework.
\end{itemize}

The third item is caused by implementation limitation can be minimized 
or removed by improving implementation details. For example, to access 
the per CPU runqueue in Linux is an optimized operation. However, the 
access to a per ox container is not so efficient, this will give a 
penalty when scheduler works inside an ox container.

There are two experiments carried out to evaluate the overhead in the 
oxc framework. In experiment A, the code execution time of frequently 
invoked oxc functions is measured. In experiment B, the overall overhead 
of oxc control is estimated through comparisons with rt throttling and 
cfs bandwidth control.

Inside each oxc function, there is a scheduling operation inside and
codes to regulate bandwidth reservation. The cost of these functions 
is the main interest in experiment A. Current oxc framework 
implementation is still a prototype. Some kernel features are not 
considered under the framework yet. For instance, the 
\texttt{priority inheritance}, which is important for the kernel's 
real-time performance and will influence number of context switches. 
So, instead of counting and analyzing context switches directly,
in experiment B the overhead of scheduling inside an ox container is
approximately evaluated by a relative way. As for the context switches 
caused by importng CBS based scheduling in the kernel, people can refer  
to \cite{Luigi09} for more iformation; yet these results do not straightly 
apply to oxc work. 

The hardware and software used in the experiment are shown in 
table \ref{tab:exp_setup}.
\begin{table}[H]%thbp]
  \centering
  \begin{tabular}{ll}\hline
	\emph{Hardware platform}\hspace{4cm}		& 	\\
	Processor			& Intel(R) Core(TM) Duo E8500	 \\
	Frequency			& 3.16GHz\\
	RAM				& 	  \\	
					&	\\	
	\emph{Software platform}\hspace{4cm}		& 	\\
	Linux distribution		& Ubuntu 11.10\\
	Compiler version		& gcc 4.6.1\\
	Kernel version			& 3.4.0-rc+ \\\hline
  \end{tabular}
  \caption{Hardware-Software platform}
  \label{tab:exp_setup}
\end{table}
The chapter is organized like this: the tracing tool and synthetic 
benchmark tool we use in the experiment are described; then it's the 
design and result analysis of each experiment; finally, what we learn
from the experiment is concluded.

\section{Ftrace in Linux kernel}
Ftrace\cite{ftrace} is an internal tracer designed to help out developers of systems to
find out what is going on inside the kernel. The name ftrace comes from
''function tracer'', which is its original purpose and the reason it is 
used here. Now there are various kinds of tracers incorporated in Ftrace.
You can use it to trace context switces, hong long interrupts are disabled,
and so on.

Ftrace uses \emph{debugfs} file system to hold control files as well as
file to display output. 
Typically, ftrace is mounted at \texttt{/sys/kernel/debug}.
\begin{lstlisting}
	#mount -t debugfs nodev /sys/kernel/debug
\end{lstlisting}
After this command, a directory \texttt{/sys/kernel/debug/tracing} will 
be created containing interfaces to configure ftrace and display results.
\begin{lstlisting}
	#cd /sys/kernel/debug/tracing
\end{lstlisting}
The following commands will be assumed to be called under \texttt{tracing}
directory.
There are several kinds of tracers available in ftrace, simply cat the
\texttt{available\_tracers} file in the \texttt{tracing} dorectory.
The output could vary with enabling or disabling kernel configuration 
options concerned with ftrace functionality in compilation time.
\begin{lstlisting}
	#cat available_tracers
	blk function_graph mmiotrace wakeup_rt wakeup function sched_switch nop
\end{lstlisting}
The \texttt{function} is the function tracer. It uses the \texttt{-pg} option
of \texttt{gcc} to have every function in the kernel call a special function
\texttt{mcount()} for tracing all kernel functions. 
The \texttt{function\_graph} is similar to the function tracer except that
the function tracer probes functions on their entry whereas the function 
graph tracer traces on both entry and exit of a function. It is called 
function graph tracer because it provides the ability to draw a graph 
of function calls similar to C code as tracing results. 
This \texttt{function\_graph} is what we use in experiments. 
To enable the function graph tracer, just echo \texttt{function\_graph} 
into the \texttt{current\_tracer} file.
\begin{lstlisting}
	#echo function_graph > current_tracer
\end{lstlisting}
A trace can be started and stopped through configuring \texttt{tracing\_on}
file. Echo 0 into this file to disable the tracer or 1 to enable it. Cat the
file will display whether the tracer is enabled or not.

The output of the trace in held in file \texttt{trace} in a human readable
format. The ftrace will by default trace all functions in the kernel. In
most cases, people only care about particular functions. To dynamically
configure which function to trace, the \texttt{CONFIG\_DYNAMIC\_FTRACE}
kernel option should be set in compilation time  to enable dynamic ftrace. 
Actually, \texttt{CONFIG\_DYNAMIC\_FTRACE} is highly recommanded and defaultly
set because of its performance enhancement. To filter which function to trace
or not, two files are used: \texttt{set\_ftrace\_filter} for enabling the
tracing of a specific function and \texttt{set\_ftrace\_notrace} to 
disable the tracing of some function. A list of available functions that you 
can add to these files is listed in \texttt{available\_filter\_functions}.
In the later experiment, the oxc function \texttt{task\_tick\_oxc} is traced
by setting up like this:
\begin{lstlisting}
	#echo task_tick_oxc > set_ftrace_filter
\end{lstlisting}


\section{Tbench}
The tbench \cite{tbench} benchmark is a tool that measures disk throughput 
for simulated netbench runs. Tbench reads a load description file called 
client.txt that was derived from a network sniffer dump of a real netbench 
run and %produces the filesystem load according to the description file. 
it produces only the TCP and process load and no filesystem calls.
One exmple to run tbench test:
\begin{lstlisting}
	$tbench_srv
	$tbench 2 -t 100
\end{lstlisting}
The \texttt{tbench\_srv} should be invoked before running \texttt{tbench}.
The second command starts two tbench connections with one client thread 
and one server thread in each connection. The two connections will run 
simultaneously and the runtime of the benchmark will be 100 seconds.

\section{Experiment A}
\subsection{The experiment design}
In this experiment, the execution time of oxc functions are measured.
In a Linux system, even if the oxc patch is applied in the kernel, 
when there is no oxc tasks, the system performs as a plain Linux system. 
In such a case, the possible oxc overheads include the code execution time 
in function \texttt{is\_oxc\_task} and the oxc related initialization when 
a scheduling group is created; both are negligible.

During the experiment, the execution time of following oxc functions are 
measured using Ftrace:
\begin{itemize} 
\item \texttt{check\_preempt\_curr\_oxc}
\item \texttt{pick\_next\_task\_oxc}
\item \texttt{put\_prev\_task\_oxc}
\item \texttt{task\_tick\_oxc}
\end{itemize} 
They are most often called oxc functions as they enclose in most frequently
invoked scheduling operations. The oxc functions like 
\texttt{enqueue\_task\_rq\_oxc} and \texttt{dequeue\_task\_rq\_oxc} only 
happen when a task enters or leaves an ox container. Inside an
ox container, enqueue and dequeue of a task still frequently happen.
However, the performance of scheduling operations of a modular
scheduler under oxc framework is not an emphasis in this experiment.

There will be six individual tests differing in the number of hyper ox 
containers in the system. In the first test, there is only one hyper 
ox container; then, in each test one more hyper ox container would be 
added. All hyper ox containers in the experiment are identical.
Each hyper ox-container has two ox containers with the same CPU 
reservation parameter $0.1ms/1ms$. Each ox container has one dummy task 
within it. The dummy task simply runs a forever while loop and it is a 
rt task with policy \texttt{SCHED\_FIFO}. This while loop task will 
exhuaust the reservation in its ox container. The scheduling policy is 
arbitrarily chosen without special thoughts. On the contrary, the 
situation for modular scheduling inside an ox container is intentionally
arranged as simple as possible to clarify the effect of oxc framework.

The measured execution time of above oxc functions comprises the time 
consumed by codes involving with the oxc control and operations defined 
in modular scheduler which are encapsulated inside the these functions.
%Here the situation for rt scheduling inside an ox container is very simple.
%And the experiment analysis can focus on oxc control overhead.

\subsection{Experiment results}
The results of six tests are listed in table \ref{tab:exp_res}.
The index of a test indicates the number of hyper ox containers in that 
test. The two fields in the pair are average vaule and standard deviation 
of the measured function execution time in micro seconds.
\begin{table}[thbp]
	\centering
	\begin{tabular}{|l||c|c|c|c|c|c|}\hline
		& \tiny{test1} & \tiny{test2} & \tiny{test3} & \tiny{test4} & \tiny{test5} & \tiny{test6}\\\hline
	\tiny{pick\_next\_task\_oxc} &\tiny{(0.168, 0.083)} &\tiny{(0.155, 0.070)} &\tiny{(0.206, 0.178)} 
							&\tiny{(0.230, 0.215)} &\tiny{(0.211, 0.216)} & \tiny{(0.246, 0.251)} \\\hline
	\tiny{put\_prev\_task\_oxc} &\tiny{(0.827, 0.049)} & \tiny{(0.834, 0.096)}&\tiny{(0.820, 0.103)} &\tiny{(0.801, 0.111)} &
					\tiny{(0.829, 0.146)} & \tiny{(0.852, 0.251)}\\\hline
	\tiny{task\_tick\_oxc} &\tiny{(0.272, 0.192)} & \tiny{(0.275, 0.182)}&\tiny{(0.263, 0.155)} & \tiny{(0.261,0.15)}& \tiny{(0.245,0.158)}& 
					\tiny{(0.249,0.146)}\\\hline
	\tiny{check\_preempt\_curr\_oxc} & - & - & - & - & - & - \\\hline
	\end{tabular}
	\caption{Measured execution time, in micro seconds, of oxc functions\\
				 \indent\hspace{4cm}in the format of \emph{(mean, standard deviation)}}
	\label{tab:exp_res}
\end{table}

The first surprise is from the row for \texttt{check\_preempt\_curr\_oxc}. 
That is, no tracing result for \texttt{check\_preempt\_curr\_oxc} is 
recorded. An analysis of overhead in this function is necessary.
The details of this function is in list \ref{check_preempt}.
There are three cases when to check if a task can preempt the currently 
running task. When only one of the two is an oxc task or both two are oxc 
tasks and not in the same ox containers, the comparison cost is just several 
\texttt{if-else} instructions. If they are two oxc tasks in the same container, 
this function follows the procedure in Linux scheduling; in addition, in our 
experiment setup, there is only one task inside an ox container. In short words, 
this function is not a significant soure for oxc framework overhead in tests. 
Although this could be a reason to explain that the ftrace fails to measure
the execution time of \texttt{check\_preempt\_cur\_oxc} during tests, it reminds
us that, given the feature of hierarchical scheduling, it may be attractive to
develop a new recording tool so as to evaluate the oxc framework more accurately.
 
The test results of other three oxc functions are illustrated in figure 
\ref{fig:pick_next}, \ref{fig:put_prev} and \ref{fig:task_tick}.
The variable parameter in each test is the number of ox containers.
And experiment results show that at least in current oxc framework, the
codes execution time is influenced by the number of ox containers in the
system.

\begin{figure}[H]%htbp]
        \centering
        \includegraphics[width=\textwidth, totalheight=0.4\textheight]{images/pick_next_task_oxc}
        \caption{Measured execution time for \texttt{pick\_next\_task\_oxc}}
        \label{fig:pick_next}
\end{figure}
\begin{figure}[H]%htbp]
        \centering
        \includegraphics[width=\textwidth, totalheight=0.4\textheight]{images/put_prev_task_oxc}
        \caption{Measured execution time for \texttt{put\_prev\_task\_oxc}}
        \label{fig:put_prev}
\end{figure}
\begin{figure}[H]%htbp]
        \centering
        \includegraphics[width=\textwidth, totalheight=0.4\textheight]{images/task_tick_oxc}
        \caption{Measured execution time for \texttt{task\_tick\_oxc}}
        \label{fig:task_tick}
\end{figure}
Figure \ref{fig:pick_next} shows the statistical result of 
\texttt{pick\_next\_task\_oxc} in each test. One observation from the figure
is that with more ox containers joining the system, the time spent on executing
the function codes fluctuates more. This trend also reflects in the
results for \texttt{put\_prev\_task\_oxc}, as in figure \ref{fig:put_prev}.
However, figure \ref{fig:task_tick} for \texttt{task\_tick\_oxc} does not 
show this pattern.

Now we are going to probe why the execution time of \texttt{task\_tick\_oxc}
is more stable than the result of the other two. A look at the body of the
three functions in list \ref{list:pick_next}, \ref{list:put_prev} and
\ref{list:task_tick}, we can find that except for the enclosed scheduling
operations, other codes in the three functions are actually the same:
to update the per CPU runqueue, to update the per container runqueue and
update the current ox container. So, the different variance behaviour in 
measured execution time for oxc functions may be affected by the performance
of sheduling operations inside oxc container. 

The three encapsulated scheduling operations \texttt{pick\_next\_task\_rt}, 
\\\texttt{put\_prev\_task\_rt} and \texttt{task\_tick\_rt} are defined in
rt scheduling class. Specifically, the codes of \texttt{task\_tick\_rt},
which is called inside oxc function \texttt{task\_tick\_oxc}, is listed 
below. This is a very simple function. If people read the other two
scheduling operations' codes in \emph{linux/sched/rt.c}, this function is 
still less complex in dealing with runqueues. When scheduling operations 
are called inside an ox container, there will be extra cost because
the raw handle of implementation details, and such an influence may 
be smaller when the scheduling operation itself is simple. This explains 
why the result in \ref{fig:task_tick} is stable in both mean value and 
standard variance.
\begin{lstlisting}[language=C,
			caption={\texttt{task\_tick\_rt}},
			label={task_tick_rt}]
static void task_tick_rt(struct rq *rq, struct task_struct *p, int queued)
{
        update_curr_rt(rq);

        watchdog(rq, p);

        /*
         * RR tasks need a special form of timeslice management.
         * FIFO tasks have no timeslices.
         */
        if (p->policy != SCHED_RR)
                return;
        /*
         * The left part has no effects in our tests.
         */
        ...

}
\end{lstlisting}



\section{Experiment B}
\subsection{Experiment design}
The aim of this experiment is to estimate the overall overhead in oxc 
framework through comparison with non real-time CPU bandwidth control 
mechanisms that are already in Linux kernel. During tests the synthetic 
load is generated by the tbench benchmark tool. The CPU bandwidth 
allocated to tbench connections are allocated by oxc control, rt 
throttling and cfs bandwidth control individually. The tbench throughput 
results of oxc framework are then compared with rt throttling and cfs 
bandwidth control. By such comparisons, the overhead introduced by oxc 
control is then evaluated in a relative way.

Two tbench connections will be set up in the system.
Each connection will be dedicated to one CPU.
Without constraints, they will consume all CPU time.
In the experiment, the CPU bandwidth allocated to tbench traffic is restricted.
The per CPU bandwidth parameter used in tests includes
$0.05ms/1ms$, $0.1ms/1ms$, $0.2ms/1ms$, $0.4ms/1ms$, $0.6ms/1ms$ and 
$0.8ms/1ms$. 
Note that these are the per CPU bandwidth that is planned to assign to 
tbench tasks. Each cpu bandwidth control mechnism will restricts the 
tbench execution not exceed the configured value. And the throughput 
results will be proportional to the overhead in each bandwidth 
control mechanism.

When rt throttling is tested, tbench clients and servers will be 
scheduled as rt tasks with policy \texttt{SCHED\_RR}. Otherwise the client 
and server in the same connection cannot run at all. Correspondingly, to 
compare with rt throttling, tbench threads inside the ox container will 
be set as rt tasks with \texttt{SCHED\_RR} policy too. 
When comparing the oxc control with cfs bandwidth control, the tbench
threads inside ox containers will run as normal tasks. 

\subsection{Experiment results}
The thrughput results are shown in table \ref{tab:expB1} and \ref{tab:expB2}.
\begin{table}[H]%thbp]
	\centering
	\begin{tabular}{|l||c|c|}\hline
		 per CPU bandwidth & rt throttling & oxc control + rt scheduling\\\hline
			0.05/1ms &	21.9335	&	18.5313	\\\hline 
			0.1ms/1ms &	43.5794 &	36.89324 \\\hline
			0.2ms/1ms &	92.7356 &	73.5099	\\\hline
			0.4ms/1ms &	172.582 &	147.806 \\\hline
			0.6ms/1ms &	233	&	225.72	\\\hline
			0.8ms/1ms &	319.297 &	297.0607	\\\hline
	\end{tabular}
	\caption{Throughputs, in Mbps/sec, under rt throttling and oxc control}
	\label{tab:expB1}
\end{table}
\begin{table}[H]%thbp]
	\centering
	\begin{tabular}{|l||c|c|}\hline
		 per CPU bandwidth & cfs bandwidth control & oxc control + cfs scheduling  \\\hline
		 0.05ms/1ms	& 24.8825	& 19.2151 \\\hline
		 0.1ms/1ms	& 52.2106	& 39.4268 \\\hline
		 0.2ms/1ms	& 106.226	& 77.4225 \\\hline
		 0.4ms/1ms	& 215.071	& 151.465 \\\hline
		 0.6ms/1ms	& 323.628	& 234.369 \\\hline
		 0.8ms/1ms	& 433.025	& 305.1 \\\hline
	\end{tabular}
	\caption{Throughputs, in Mbps/sec, under cfs bandwidth and oxc control}
	\label{tab:expB2}
\end{table}
After a glance on the two tables, it's apparent to note that the 
throughput results under cfs bandwidth control outperform the other two. 
There are two reasons for this. Firstly, the overhead brought by cfs 
bandwidth control is indeed lower than the other two mechnanisms.
Secondly, both oxc tasks and rt tasks are more greedy than cfs tasks when 
they occupy a CPU. For oxc tasks, both rt and normal tasks are lower
priority tasks, between which normal tasks are lower priority tasks.
the oxc task or rt tasks will not give up a CPU to a lower priority
task until the CPU reservation is totally consumed. So lower priority 
tasks in the system are stressed more under oxc control or rt throttling, 
especially when high reservation parameter is configured. However, in cfs 
scheduling, cfs tasks, with or without CPU reservation, can share the CPU 
evenly.

\begin{figure}[htbp]
        \centering
        \includegraphics[width=\textwidth,totalheight=0.4\textheight]{images/expB1}
        \caption{\emph{oxc control} vs. \emph{rt throttling}}
        \label{fig:expB1}
\end{figure}
Having solved the above question, let's first study the peformance of tbench 
tasks in oxc framework when they are scheduled with \texttt{SCHED\_RR} 
policy. The comparison between oxc control and rt throttling in 
table \ref{tab:expB1} is visulized in figure \ref{fig:expB1}.
At first, the throughput result under rt throttling is higher and 
growing faster than the result in oxc control. However, as the more CPU 
bandwdith is reserved to tbench tasks, the throughtput results under the 
two means are converging. In fact, the throughtput growing trend in oxc 
control are consistent. When it comes to rt throttling, it's not like this.
When a relatively small fraction of CPU is assigned to rt throttling
and oxc control, rt throttling shows better performance. However, with
increasing the reserved CPU bandwdith, the stress of rt throttling on
the whole system is rising too, which slows growth of the throughput. 
Under oxc control, the throughput result is almost linear with the 
given reserved bandwidth. The overhead in oxc control behaves as 
a constant factor.

The comparason between oxc control and CPU bandwidth control is ahown 
in figure \ref{fig:expB2}. As we just analyzed, cfs bandwidth control has
much better throughput result. One observation is that althouth with
less rasing speed, the throughput increasing trend in oxc control has the
similar shape as in cfs bandwidth control. 
The result under oxc framework has another meaningful implication. 
When oxc control is used, allocations of bandwidth in the system should be
cautioned so as to achieve an optimal system performance. 

\begin{figure}[htbp]
        \centering
        \includegraphics[width=\textwidth, totalheight=0.4\textheight]{images/expB2}
        \caption{\emph{oxc control} vs. \emph{cfs bandwidth control}}
        \label{fig:expB2}
\end{figure}
\begin{figure}[htbp]
        \centering
        \includegraphics[width=\textwidth, totalheight=0.4\textheight]{images/expB3}
        \caption{\emph{oxc control + rt throttling} vs. \emph{oxc control + cfs scheduling}}
        \label{fig:expB3}
\end{figure}
At last, figure \ref{fig:expB3} mixes the statistics in table \ref{tab:expB1}
and \ref{tab:expB2} and draws the throughput results under oxc control with
rt throttling and cfs scheduling in the same graph.
The results are quite close, and the difference can be regarded as
the scheduling cost between rt and cfs scheduling in an ox container.
Still, cfs scheduling shows better results than rt scheduling even under 
oxc framework. This says that cfs scheduling introduces less overhead in the
system than rt scheduler. Such a comparison causes us to think about one
possible application of oxc framework. In some cases, when to compare two
schedulers, we can set up the environment inside an ox container; or by preparing
a certain number of ox containers, a lightweight networked testbed. 

\section{Experiment feedbacks}
The experiment does not show that the performance of oxc control is better than
existing bandwidth control methods. This is also not the experiment objective( the
comparison result with rt throttling is a small surprise). The experiment outcome
has an unstated meaning for future development of oxc framework.
Experiment A gives precise measurement of oxc function execution time and 
confirms the necessary to improve implementation quality of the oxc framework.
In experiment B, the overall performance of oxc framework is compared with
rt throttling and cfs bandwidth control. Its experiment analysis show us that
how to distribute CPU bandwidth will affect both the work inside an ox container
and the whole system behaviour; it also raises one example for oxc control usage. 
These feedbacks would be incorporated in the evolvement of the oxc framework.

\chapter{Conclusions and Future Work\label{chap:con}}



%%% Local Variables: 
%%% mode: latex
%%% TeX-master: "main"
%%% End: 

\backmatter
\chapter{Acknowledgments}
After six months' efforts, I am concluding this thesis,
which is the most devoted work I have ever done. Great 
appreciation to my thesis supervisor, Giuseppe Lipari.
It's amazing to work with you. Your involvment keeps
delicate balance that the problems I meet (in thesis 
and life) can be solved and my potential for inpdendent 
and creative thinking is highly motivated. There are too 
many stupid, unrelated and weird opinions that I have 
told you. However, I could always get positive and 
beneficial feedbacks. 
Thanks to my supervisor Luigi Palopoli in Trento.
At the early days after I came to Italy, I was confused. 
You helped me be confident. 
I enjoyed the last two years' time. I met these professors, 
colleagues and friends that make my life special. I am not 
going to list every name. I sincerely thank you all. 
Dear Liang, because of you, I am the luckiest person in 
the world. 
Baba mama, I love you!


\bibliographystyle{ieeetr}%{plain}
\bibliography{biblio}

\end{document}

%%% Local Variables: 
%%% mode: latex
%%% TeX-master: t
%%% End: 
