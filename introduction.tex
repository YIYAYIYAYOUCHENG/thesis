\chapter{Introduction\label{chap:introduction}}

%%
%% PEPPE: I did some more re-writing, splitted the chapter into sections
%%

Linux is the most widely deployed open source GPOS. Linux flavors are 
used in a very wide range of systems and application area, from servers 
for cloud computing, to desktop computing, to tablet and smart phones, 
to industrial embedded systems. 

One critical part of every Operating System is the task scheduler. For
Linux, the scheduling system is also very critical given the wide
range of application scenario in which this system is used. In facts,
every scenario has different requirements and there is no single
scheduler that can optimally satisfy them all.

Traditionally, one of the aim of Linux scheduling is to distribute CPU
cycles fairly among tasks and task groups according to their relative
importance. Another important requirement is to make \emph{interactive
  applications} reactive to user input. There is no precise definition
of an \emph{interactive task}: the kernel uses heuristics to identify
interactive tasks from batch processing tasks, in order to give higher
priority to interactive tasks, thus reducing their response time.

One important class of applications consists of hard real-time or soft
real-time systems. In order to work well, these applications have
requirements in terms of the CPU cycles they receive, or on the time
they need to react to external events. Hard and soft real-time
applications needs to be guaranteed prior to their execution that they
will be able to meet all their timing requirements (if hard real-time)
or most of their timing requirements (if soft real-time).
Unfortunately, such timing guarantee (or \emph{real-time guarantee})
cannot be provided just by using an heuristic scheduler. 

Examples of soft real-time applications are multi-media applications. 
%
% TODO: Please spend some time describing what is a multi-media
% application, and how it is different from a interactive application.
% DONE
Multimedia finds its application in various areas including DVD player,
video conference, virtual reality, computer games, etc. Multimedia 
activities include video and audio streams, which are the challenge 
for the scheduler in a GPOS. Because such media streams are characterized 
by an implicit timing requirement. More specifically, video/audio streams 
in a computer are digitalized and divided into frames. In order to 
reproduce the stream, the sequence of frames must be processed 
respecting some temporal constraints. As an example, a rate of 25 frames 
per second (fps) is allowed in MPEG-2 video, which means that every 40ms 
a frame is expected to be loaded, decoded, filtered and played. If a frame 
is lost or played too late, the user may not notice any degradation. But 
if the skipped frames or frame delays are over some threshold, users start 
perceiving a certain degradation on the quality of the service. In Linux, 
a multimedia application is normally treated as interactive tasks. Even 
through, the scheduler has no concept of the temporal requirement for a 
stream and there is no way to guarantee some degree of quality for a 
multimedia application.    

\section{CPU reservations}
 
%
% TODO: Here it is necessary to introduce resource reservations,
% as a way to manage bandwidth. 
% Key concepts are:
% - Open Systems (tasks can be dynamically launched and removed from the system)
% - There is the need to isolate tasks by reserving bandwidth
% - There is the need to perform admission control, to not overload the system
%
% One you do this, you can discuss about what is the best way to
% distribute bandwidth (i.e. the need to do it hierarchically)
% DONE 
The temporal constraint of multimedia applications can be solved by
a class of \emph{resource reservation} (RR) algorithms in real time 
scheduling literature. The feature of a RR algorithm is temporal 
isolation. That is, a fraction of CPU time can be reserved to a set
of tasks. By applying resource reservation in Linux, multimedia
applications can meet their timing requirements. Our work exploits
the RR mechanism for a more general use in managing CPU power
distribution. The RR algorithm we use is CBS~\cite{AbeniB98}. More 
on it will be introduced in section~\ref{sec:CBS}.

The following example shows a scheme of CPU power distribution.
In many cases, people prefer to distributing cpu power in a privileged and 
predictable way. For instance, to give 10\% of the total CPU cycles to a 
set of tasks; furthermore, take half of this 10\% and assign it to a 
subset of the tasks.
\section{Multi-core systems}

Nowadays, multi-core architectures are successfully used to boost
computation capability for computing devices. Common computing systems
with multiple processors are becoming mainstream: it is no surprise to
see an embedded device with more than one CPU inside. So, it is 
necessary for a CPU power distribution scheme to support multi-core 
platforms. Additionally high speed processors and more cores rather 
than solving the CPU timing guarantee problem do bring more challenges 
for the developer. Also, no matter how powerful the platform is, there 
is always time when people need more. These are all considerations 
when design the CPU power management work.
%in a real-time way.
%And the last golden rule for a GPOS CPU power management mechanism is:
%\texttt{support multi-processor platform}.
%
% TODO: 
% The last thought is interesting, needs to be further elaborated.
% DONE


\section{The Linux scheduler}

%
% TODO: This ought to be explained in greater detail. You can remand
% to the appropriate chapter (1.2), and make some little anticipation
% here.
% DONE
In mainline Linux kernel, there are two so called real time (RT)
scheduling policies: SCHED\_FIFO and SCHED\_RR. Tasks scheduled by 
them are called ``real time (RT) tasks''. They are required in POSIX 
standard for POSIX-compliant operating systems like Linux. 
Unfortunately, despite of the name, these two policies can only 
provide real time gurantee in very limited conditions. In mainline 
Linux, there exist two non real time mechanisms to control CPU 
bandwidth distribution: real time (RT) throttling and complete 
fairness scheduler (CFS) bandwidth control. In principle, they 
are the same technique working for different types of tasks. 
For more on Linux scheduling, please read chapter 
\ref{chap:background}.


\section{Contributions of this thesis}

This thesis will present the design and development of a hierarchical 
multi-processor reservation mechanism in Linux that is not constrained 
with specific scheduling algorithms implemented in the kernel.
This work is started from scratch.

%
% TODO: Here you must first list what you have done. 
% DONE
Before discussing details on our work, let's think a little bit what
is the good mechanism to distribute the CPU power in a GPOS, taking 
Linux as the study case. 
\begin{itemize}

\item   {\texttt{Compatibility}}\\
	In principle, a scheduler does distribute CPU time. There are 
	several schedulers in Linux and each deals with its own type 
	of tasks. So, the constraint for a scheduler to manage CPU 
	power is that it can only handle a subset of tasks. A general 
	mechanism that is compatible with different shedulers are better.  
\item 	{\texttt{Hierarchical distribution}}\\
	Tasks in the Linux system are organized in a flat or hierarchy, 
	depending on the kernel compilation option. In modern systems, the 
	hierarchical way is more preferred. The recommended method should 
	be able to hierarchically distribute the CPU power.
\item 	{\texttt{Predictability}}\\
	Because there are various kinds of tasks inside Linux, among which
	some require (hard or soft) timing guarantee and some do not. 
	The CPU power distribution should be predictable. When it comes to 
	predictability, a guarantee test or admission control is always 
	necessary since the predictability can only be available under 
	a condition. For instance, the CPU utilization is no larger than 1. 
\item	{\texttt{Support multi-procressor platforms}}
\end{itemize}
In this article, the above four items are called \emph{golden rules} 
for CPU power distribution design in the GPOS. We design and develop 
a RR based framework that fulfils all golden rules. 
We call the framework Open-Extension Container (OXC) scheduling
framework. The Open-Extension (OX) Container is a new data structure in
Linux that is introduced by our work.  It is the fundamental element
in the framework. More on an ox container is in chapter~\ref{chap:design}. 
In oxc scheduling framework, each ox-container can reserve an amount of 
bandwidth from a CPU through CBS rules. Several ox containers can work
together and reserve CPU bandwidths from multiple processors. From the 
Linux scheduler's point of view, an ox container is regared as a virtual 
CPU. This is why the reserved bandwidth through an ox container can be 
utilized by different kinds of tasks in Linux. Details on the development 
of the oxc framework is in chapter~\ref{chap:impl}.

The oxc framework has two features:
\begin{itemize}
\item It can predictably distribute the cpu cycles of a
  multi-processor platform to a set of tasks and its subsets, without
  requirements for details how they are scheduled.
\item Under the framework, scheduling policies can be applied in a
  fine-grained way.
\end{itemize}
This first feature is actually a conclusion of four golden rules.
The second feature has nothing to do with CPU time distribution.
Usually, when a scheduling algorithm is applied in a system, it is
indeed applied in the whole system level. One by-product of our 
framework is that the working scale of a scheduling algorithm can
be constrainted in part of the system. The basic unit to apply a 
scheduling policy under oxc framework is an ox container.

%
% TODO: I (still) need to review this last part. 
% Some of the things should be moved elsewhere.
%
%In Linux, there is a scheduling system on each CPU. Different such per
%CPU scheduling systems construct the system level scheduling by task
%migration mechanisms among different CPUs. For the first time in
%Linux, our framework provides the opportunity to build extra
%scheduling systems besides these destinated with each CPU. We call the
%framework Open-Extension Container (OXC) scheduling
%framework. Open-Extension (OX) container is a new data structure in
%Linux that is introduced by our work.  It is the fundamental element
%in the framework. Based on ox container structure, the concept ``per
%oxc scheduling system'', whose behaviour is the same as ``per CPU
%scheduling system'' in Linux, is given. Several per oxc scheduling
%systems cooperate and work as the ``pseudo (Linux) system level
%scheduling''.
%
%In oxc scheduling framework, each ox-container can reserve an amount
%of bandwidth from a CPU through CBS rules \cite{AbeniB98}. The per oxc
%scheduling system based on it utilize this computing power to shcedule
%tasks as if working on a less powerful cpu. This is how oxc framework
%distribute reserved CPU cycles to tasks under it. Because the per oxc
%scheduling has the same behaviour as per CPU scheduling in Linux,
%general types of tasks can run using the reserved bandwidth. On
%multiple processor platforms, different OXCs can inpdependently
%reserve bandwidths from a subset of total CPUs and scheduling systems
%above them work together to imitate the behaviour of the Linux system
%level scheduling. The basic unit to apply a scheduling policy under
%OXC framework is an OX container.

\section{State of the Art}

There is work that extends Linux with real time capabilities to fulfil 
the timing guarantee requirement: RTAI~\cite{rtai}, AQuoSA~\cite{Luigi09}, 
\texttt{sched\_deadline} patch~\cite{Dario09}, 
IRMOS real-time framework~\cite{irmos}, RESCH~\cite{resch}, etc.
%
% TODO: Do not forget to insert citations where necessary, 
% IRMOS, RESCH, RT-Xen.
% DONE
Instead of modifying the system directly, RT-Xen~\cite{rtxen} tries to 
apply real time mechanisms in the hypervisor level(Xen). 

Each work has its emphasis. Initially, our work is motivated by
IRMOS. The IRMOS can meet three of our golden rules except for 
the compatibility one. IRMOS scheduler can only reserve CPU 
bandwidth for RT tasks in Linux.  

RTAI adapts a dual kernel approach. Besides the Linux kernel, there is
a real time kernel for serving real time tasks. It aims to reduce the
wost case latency to run a real time task. As for the latency, the 
performance of RTAI is a benchmark for other work. 

In AQuoSA, the CBS rules are used to reserve CPU reservations. 
As a CPU time distribution mechanism, it is based on RT scheduling 
with \texttt{SCHED\_RR} policy and fails the multi-processor support 
rule. However, inside the AQuoSA architecture, there is a feedback-based 
mechanism to dynamically cnfigure the rservation parameter. Such an 
adaptive resource allocation is meaningful since workloads in a system 
will fluctuate with time. 

The \texttt{sched\_deadline} patch is very promising to be merged 
into the mainline kernel as the deadline based scheduler. 
It should be expected that the oxc framework can also reserve 
bandwidths for tasks scheduled by this deadline based scheduler. 

RESCH implements several real-time algorithms that can be
loaded into the kernel as modules. This reminds us if we can realize 
the oxc framework as modules. 

RT-Xen tries to bridge the gap between real-time scheduling theory and 
Xen hypervisor and insantiates a suite of real-time scheduling algorithms. 
To provide real-time guarantee in the cross platform hypervisor layer is 
very attractive. And very likely the real-time support from the operating 
system is necessary for such a purpose. In this case, a real-time CPU 
bandwidth control mechanism like our work will be helpful.

Our work is purely interested in CPU bandwidth distribution.
To the author's knowledge, there does not exist other 
CPU power management work that can achieve all golden rules.

%
% TODO This section needs to be completed with a more extensive list
% and description of the state of the art, with comparison with OXC
% (i.e. in what OXC is different, or even superior).
% DONE
\section{Organization of the thesis}
Chapter~\ref{chap:background} ({\bf Background}) gives a brief overview
of the knowledge that is needed to understand the oxc framework. At first,
the CBS algorithm for resource reservation. Then the architecture of 
Linux scheduling is described. By architecture it means the emphasis is 
on how different scheduling components are organized. In the end of this
chapter, related work is generally discussed, including the RT throttling
and CFS bandwidth control in mainline Linux and a more extensive 
introduction on some work listed above.

Chapter~\ref{chap:design} ({\bf Design}) gives the difinition of an ox 
container and its attributes. The theory of ox container scheduling, which
is the soul of our work, is explored in this chapter. 

Chapter~\ref{chap:impl} ({\bf Implementation}) presents details of the oxc
framework. It first explains new data structures added in the kernel and 
extensions on original kernel structures. The next part concerns how to 
implement the oxc scheduling theory in the kernel. Then, important functions,
including how to reserve CPU bandwidths, in the oxc framework are studied.
A complete introduction on interfaces of the framework and how to use them
is also in this chapter. At last, the relationship between oxc framework
and Linux CPU power management methods is investigated.

In Chapter~\ref{chap:exp} ({\bf Experiment}), two experiments are conducted
to evaluate the overhead and performance of the oxc framework (or oxc control).
The experiment set up, kernel recording tool, benchmark tool,
etc, are introduced first. Then at each experiment, there is the design, 
data collection and result analysis.

Finally, the Chapter~\ref{chap:con} ({\bf Conclusion}) sums up the current
status of the oxc framework and discusses future development of the work. 
%

%%% Local Variables: 
%%% mode: latex
%%% TeX-master: "main"
%%% End: 
