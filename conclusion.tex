\chapter{Conclusions and Future Work\label{chap:con}}
In this thesis we present the design and development of the oxc scheduling
framework. The oxc framework is a hierarchical multi-processor reservation 
mechsnism that targets to manage CPU power distribution to tasks in Linux
without constraints on the scheduling details. 

Alongside the design and development of the framework, we bring some
new ideas into the world. We conclude the four golden rules that may
be useful in developing and evaluating a CPU power management 
mechanism for a GPOS. We propose the ox container structure in Linux
and the oxc scheduling theory. We formalize the relationship among
different components in Linux scheduling by inventing the concept
scheduling route, which will assist people for studying and probing
the Linux kernel scheduling.

The theoretical background of the oxc framework is the oxc scheduling,
which may be have a more profound influence besides our framework. 
In the world of oxc scheduling, tasks are managed by a scheduler around 
a virtual CPU. An independent scheduling system can be constructued 
above the virtual CPU whose behaviour is the same as a Linux per CPU
scheduling system. And the theory of oxc scheduling can find its use in
any case when an independent scheduling is needed. A big feature of the 
oxc scheduling is that it is naturally compatible with Linux schedulers. 

Although current implementation of the oxc framework is still a prototype,
it shows great potential for an opportunity to be seriously applied in real
world. For the first time, there is work in Linux that can manage CPU power
distribution for all types of tasks. It can reserve bandwidths for RT tasks
and CFS tasks in Linux, which demonstrates the theory of oxc scheduling. 
Under the framework, CFS task group scheduling is support.
Naivly work on multi-processor support and hierarchical reservation 
distributing are included in the framework and demonstrate that under 
the oxc famework such functionalities are achievable.  

During experiments, the overhead of oxc functions is measured and 
confirms that the there is the necessity to refine the implementation
of the oxc scheduling control. In another experiment, the performance of
the oxc famewrok is not as good as existing work; however, it also shows
the feasibility of the framework in Linux.

In order for the oxc framework to really attract interests from both 
academic and industrial fields. There are four issues that must be 
sloved first.
\begin{itemize}
\item The CPU bandwidth allocation in the oxc framework now is made in 
a uncontrolled way depending on the wise decision from users. A admission
control for bandwidth distribution should be added.
\item Strategies for task migrations among ox containers using the same 
reservation, which means ox containers in the same hyper ox container,
should be added. 
\item Temporarily, the hierarchical distribution of reserved CPU bandwidth
is implemented in a faked way. That is, the bandwidth distributed to sub
groups is a new reservation from the system instead of inheriting from 
the parent's reservation. The connection between parent's and children' 
reservations needs to be added. 
\item Until now, the oxc framework does not have a consideration for 
resource sharing among ox containers. If two tasks accessing the same
critical section are in the same ox container, the Linux has its way
to schedule the two tasks. However, if the two are in different ox 
containers or one is oxc tasks and the other one is not, it may happen
that an earliest deadline ox container is blocked. Now the oxc framework 
has no work on this problem. For the similiar reason, synchronization
between tasks from different ox containers is a dangerous operation.
\end{itemize}
Besides the above four items related to the survival of the oxc work, 
there are some less critical issues that are better to be done.
 
Under oxc framework, the performance of RT throttling is not satisfiable.
The relationship between oxc framework and \texttt{cpuset} cgroup 
subsystem is not well studied. According to the oxc scheduling theory,
a scheduler's influential scale can be flexibly controlled. Now in oxc 
framework, a scheduling class works without restraints. It is greatly 
meaningful to explore the oxc control together with other work like AQuoSA, 
RT-Xen, RESCH, etc. 

%%% Local Variables: 
%%% mode: latex
%%% TeX-master: "main"
%%% End: 
