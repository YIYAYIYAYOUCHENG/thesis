\chapter{Experimental Results\label{chap:exp}}
The overhead introduced by oxc framework composes two parts:
\begin{itemize}
\item The time required to execute the code of framework functions.
\item The context switches introduced by the oxc framework.
\end{itemize}

Current oxc framework implementation is still a prototype one. Some kernel 
features are not considered under the framework yet. For example, the 
\texttt{priority inheritance}, which is important for the kernel's real time
performance and will influence number of context switches. 
However, the execution time of oxc function codes is fixed and can be measured.
As for the context switches caused by importng CBS based scheduling in the
kernel, there is previous work .... This may help readers understand what 
happens under the oxc framework. In the rest of this chapter, our experiment
will concentrate on measuring execution time of oxc functions.

\section{Ftrace in Linux kernel}
Ftrace is an internal tracer designed to help out developers of systems to
find out what is going on inside the kernel. The name ftrace comes from
''function tracer'', which is its original purpose and the reason it is 
used here. Now there are various kinds of tracers incorporated in Ftrace.
You can use it to trace context switces, hong long interrupts are disabled,
and so on.

Ftrace uses \emph{debugfs} file system to hold control files as well as
file to display output. 
Typically, ftrace is mounted at \texttt{/sys/kernel/debug}.
\begin{lstlisting}
	#mount -t debugfs nodev /sys/kernel/debug
\end{lstlisting}
After this command, a firectory \texttt{/sys/kernel/debug/tracing} will 
be created containing interfaces to configure ftrace and display results.
\begin{lstlisting}
	#cd /sys/kernel/debug/tracing
\end{lstlisting}
The following commands will be assumed to be called under \texttt{tracing}
directory.
There are several kinds of tracers available in ftrace, simply cat the
\texttt{available\_tracers} file in the \texttt{tracing} dorectory.
\begin{lstlisting}
	#cat available_tracers
	blk function_graph mmiotrace wakeup_rt wakeup function sched_switch nop
\end{lstlisting}
The \texttt{function} is function tracer. It uses the \texttt{-pg} option
of \texttt{gcc} to have every function in the kernel call a special function
\texttt{mcount()} for tracing all kernel functions and measure execution time 
of them.  This is what we need. To enable the function tracer, just \emph{echo} \texttt{function} into the \texttt{current\_tracer}
file.
\begin{lstlisting}
	#echo function > current_tracer
\end{lstlisting}
The trace can be started and stopped through configuring \texttt{tracing\_on}
file. Echo 0 into this file to disable the tracer or 1 to enable it. Cat the
file will displat whether the tracer is enabled or not.

The output of the trace in held in file \texttt{trace} in a human readable
format. The ftrace will by default trace all functions in the kernel. In
most cases, people only care about particular functions. To dynamically
configure which function to trace, the \texttt{CONFIG\_DYNAMIC\_FTRACE}
kernel option should be set in compilation time  to enable dynamic ftrace. 
Actually, \texttt{CONFIG\_DYNAMIC\_FTRACE} is highly recommanded and defaultly
set because of its performance enhancement. To filter which function to trace
or not, two files are used, one for enabling and one for disabling the 
tracing of specific functions. They are \texttt{set\_ftrace\_filter} and 
\texttt{set\_ftrace\_notrace}. A list of available functions that you can add
to these files is listed in \texttt{available\_filter\_functions}.

\section{Experiment set up}

\begin{table}[hbp]
  \centering
  \begin{tabular}{ll}\hline
	Hardware platform\hspace{4cm}		& 	\\
	Processor			&	 AMD 	\\
	Frequency			& 1666MHz\\
	RAM size			& 512 Mb \\\hline
    %First row &  centred   &  .7  \\\hline\hline
  \end{tabular}
  \caption{Hardware-Software platform}
  \label{tab:mytable}
\end{table}

